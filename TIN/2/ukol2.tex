\documentclass[a4paper,12pt]{article}

\usepackage[utf8]{inputenc}
\usepackage[czech]{babel}
\usepackage[IL2]{fontenc}
\usepackage[left=3cm,text={15cm, 23cm},top=3.5cm]{geometry}

\usepackage{graphicx}
\usepackage{tikz}

\usepackage{amssymb}
\usepackage{amsmath}
\usepackage{amsthm}
\usepackage{forest}
\usepackage[ruled,czech,linesnumbered,noline]{algorithm2e}
\usepackage{array}
\usepackage{multirow}
\usepackage{textcomp}

\usepackage{xpatch}
\xpretocmd{\algorithm}{\hsize=\linewidth}{}{}
\xpretocmd{\procedure}{\hsize=\linewidth}{}{}

\usepackage{enumitem}
%\usepackage[neverdecrease]{paralist}

\usepackage{pgf}
\usepackage{tikz}
\usetikzlibrary{arrows,automata}

\newcounter{counten}
\setcounter{counten}{1}

\title{Úkol č. 2 do předmětu Teoretická informatika}
\author{Vojtěch Havlena (xhavle03)}
\date{}

\begin{document}

\maketitle

\setlist[enumerate,1]{leftmargin=0.2cm}

\begin{enumerate}[label=\textbf{\arabic*}.]
 \item {\bfseries Příklad}
  \begin{itemize}
   \item[--] Předpokládejme, že jazyk $L$ je bezkontextový. Potom podle Pumping lemma pro bezkontextové jazyky 
    existuje konstanta $k > 0$ taková, že $\forall z\in L \wedge |z| \geq k \Rightarrow z = uvwxy \wedge 
    vx\neq\varepsilon \wedge |vwx| \leq k \wedge uv^iwx^iy\in L$ pro každé $i \geq 0$.
    
   \item[--] Zvolme libovolné takové $k > 0$, splňující výše uvedené. Dále zvolme řetězec $z = a^kb^kc^kd^k \in L$, přičemž
    $|z| = 4k \geq k$.
    
   \item[--] Z řetězců $u, v, w, x, y \in\Sigma^*$, které splňují Pumping lemma zvolme libovolné z nich 
   ($z = uvwxy$, $|vwx| \leq k$ a $vx\neq\varepsilon$). Libovolný výběr těchto řetězců musí spadnout do nějaké z následujících kategorií.
   \begin{enumerate}
    \item $vwx \in \{a\}^+$: Vzhledem k tomu, že $vx\neq\varepsilon$, tak pro $i \geq 2$  řetězec 
      $z' = uv^iwx^iy \notin L$, protože počet symbolů $a$ v řetězci $z'$ je alespoň o 1 větší, než je počet symbolů $c$ v řetězci $z'$.
    \item $vwx \in \{a\}^+.\{b\}^+$: V případě, kdy $v$ či $x$ obsahuje symboly $a$ i $b$, řetězec $z' = uv^iwx^iy \notin L$ pro $i\geq 2$, protože 
      v řetězci $z'$ bude více alternací mezi podřetězci. V případě, kdy $v$ i $x$ jsou tvořeny buď pouze symboly $a$ nebo pouze symboly $b$, tak 
      pro $i \geq 2$ řetězec $z' = uv^iwx^iy \notin L$, protože buď je počet symbolů $a$ v $z'$ větší jak počet symbolů $c$ v $z'$, nebo je počet 
      symbolů $b$ v $z'$ větší jak počet symbolů $d$ v $z'$ (popřípadě oba zároveň). 
      (Zvýšil se počet symbolů $a$ nebo $b$, ale počet symbolů $c$ a $d$ v $z'$ zůstává stejný).
    \item $vwx \in \{b\}^+$: Opět pro $i \geq 2$ řetězec 
      $z' = uv^iwx^iy \notin L$, protože počet symbolů $b$ v řetězci $z'$ je alespoň o 1 větší, než je počet symbolů $d$ v řetězci~$z'$.
    \item $vwx \in \{b\}^+.\{c\}^+$: Opět v případě, kdy $v$ či $x$ obsahuje symboly $b$ i $c$, řetězec $z' = uv^iwx^iy \notin L$ pro $i\geq 2$, protože 
      v řetězci $z'$ bude více alternací mezi podřetězci. V případě, kdy $v$ i $x$ jsou tvořeny buď pouze symboly $b$ nebo pouze symboly $c$, tak 
      pro $i \geq 2$ řetězec $z' = uv^iwx^iy \notin L$, protože buď je počet symbolů $b$ v $z'$ větší jak počet symbolů $d$ v $z'$, nebo je počet 
      symbolů $c$ v $z'$ větší jak počet symbolů $a$ v $z'$ (popřípadě oba zároveň). 
      (Zvýšil se počet symbolů $b$ nebo $c$, ale počet symbolů $a$ a $d$ v $z'$ zůstává stejný).
    \item $vwx \in \{c\}^+$: Pro $i \geq 2$ řetězec 
      $z' = uv^iwx^iy \notin L$, protože počet symbolů $c$ v řetězci $z'$ je alespoň o 1 větší, než je počet symbolů $a$ v řetězci~$z'$.
    \item $vwx \in \{c\}^+.\{d\}^+$: Opět v případě, kdy $v$ či $x$ obsahuje symboly $c$ i $d$, řetězec $z' = uv^iwx^iy \notin L$ pro $i\geq 2$, protože 
      v řetězci $z'$ bude více alternací mezi podřetězci. V případě, kdy $v$ i $x$ jsou tvořeny buď pouze symboly $c$ nebo pouze symboly $d$, tak 
      pro $i \geq 2$ řetězec $z' = uv^iwx^iy \notin L$, protože buď je počet symbolů $c$ v $z'$ větší jak počet symbolů $a$ v $z'$, nebo je počet 
      symbolů $d$ v $z'$ větší jak počet symbolů $b$ v $z'$ (popřípadě oba zároveň). 
      (Zvýšil se počet symbolů $c$ nebo $d$, ale počet symbolů $a$ a $b$ v $z'$ zůstává stejný).
    \item $vwx \in \{d\}^+$: Pro $i \geq 2$ řetězec 
      $z' = uv^iwx^iy \notin L$, protože počet symbolů $d$ v řetězci $z'$ je alespoň o 1 větší, než je počet symbolů $b$ v řetězci~$z'$.
   \end{enumerate}
   \item[--] Pro žádný výběr řetězců $u, v, w, x, y$ nejsme schopni dosáhnout toho, aby $z = uvwxy \wedge vx \neq\varepsilon \wedge |vwx| \leq k \wedge \forall i \geq 0: uv^iwx^iy\in L$, což je spor.
   Jazyk $L$ tedy není bezkontextový.
  \end{itemize}
 
 
 \item {\bfseries Příklad}
 
  V celém příkladu pracuji s následujícím vyjádřením rozdílu množin: $L\setminus L' = \overline{L'}\cap L$.
  Dále pokud nebude řečeno jinak, všechny zmíněné věty pocházení z opory k předmětu TIN.
  \begin{enumerate}
   \item $L \in\mathcal{L}_3, L' \in\mathcal{L}_3$: Podle věty 3.23 jsou regulární jazyky uzavřeny vůči průniku
    i doplňku, tedy i $\overline{L'}\cap L = L\setminus L'$ bude regulární jazyk a tím pádem i bezkontextový jazyk.
    
   \item $L \in\mathcal{L}_3, L' \in\mathcal{L}_2$: Podle věty 4.24 bezkontextové jazyky nejsou uzavřeny vůči doplňku
    a tedy $\overline{L'}$ není nutně bezkontextový jazyk. Nyní předpokládejme, že $L, L'$ jsou jazyky nad abecedou $\Sigma$, potom stačí zvolit $L = \Sigma^* \in\mathcal{L}_3$, 
    tedy $L\setminus L' = \overline{L'}\cap L = \overline{L'}$, což podle předchozí věty není nutně bezkontextový jazyk. A tedy $L\setminus L'$ není
    nutně bezkontextový jazyk. 
    
   \item $L \in\mathcal{L}_2, L' \in\mathcal{L}_3$: Podle  věty 3.23 jsou regulární jazyky uzavřeny vůči doplňku, tedy 
    $\overline{L'}\in\mathcal{L}_3$. Dále podle věty 4.22 jsou bezkontextové jazyky uzavřeny vzhledem k průniku s regulárními
    jazyky, tedy $\overline{L'}\cap L = L\setminus L'$ je nutně bez\-kon\-textový jazyk.
    
   \item $L \in\mathcal{L}_2, L' \in\mathcal{L}_2$: Podle věty 4.24 bezkontextové jazyky nejsou uzavřeny vůči doplňku a průniku.
    Jazyk $\overline{L'}$ tedy není nutně bezkontextový. Nyní opět předpokládejme, že $L, L'$ jsou jazyky nad abecedou $\Sigma$, potom stačí zvolit $L = \Sigma^* \in\mathcal{L}_2$, 
    tedy $L\setminus L' = \overline{L'}\cap L = \overline{L'}$, což podle předchozí věty není nutně bezkontextový jazyk. A tedy 
    $L\setminus L'$ není nutně bezkontextový jazyk.
  \end{enumerate}

  \enlargethispage{1em}
 \item {\bfseries Příklad}
 \begin{enumerate}
  \item Algoritmus pro výpočet minimální váhy je založen na iterativním výpočtu minimální váhy řetězce, který lze vyderivovat 
  z každého neterminálního symbolu v gramatice $G$, převedenou do Chomského normální formy. Tato minimální váha pro každý neterminál je označena jako $M_A$, $A\in N$. 
  
  Hlavní myšlenka algoritmu je následující: Na začátku je pro každý neterminál nastavena hodnota $M_A := \infty$. Pro pravidla gramatiky ve tvaru $A\rightarrow x$, $x\in(\Sigma\cup\{\varepsilon\})$, je možné $M_A$ spočítat přímo jako $M_A = \min(M_A, ||x||)$.
  Jakmile již došlo k nastavení hodnoty $M_A < \infty$, tak je možné uvažovat i pravidla, která obsahují na pravé straně právě tyto neterminály, 
  které již mají předběžně spočítanou minimální váhu. Potom tedy pro pravidla tvaru $A \rightarrow BC$, $A, B, C \in N$, se $M_A$ spočítá jako $M_A = \min(M_A, M_B + M_C)$.
  V případě, že je hodnota $M_A$ aktualizována (tj. nastavena na menší váhu), tak je nutné znovu přepočítat váhu neterminálů, které vystupují 
  na levé straně pravidlech obsahující $A$ na pravé straně pravidla. Tedy pokud aktualizuji hodnotu $M_A$ je nutné přepočítat váhu všech neterminálů $B$, takových, že 
  $B\rightarrow AX$ nebo $B\rightarrow XA$, $X,B,A \in N$, jsou pravidla z $P$ (toto zajišťuje procedura Aktualizuj).
  Tento postup se opakuje, dokud minimální váha není spočítána pro každý neterminál. Potom výstupem algoritmu je hodnota $M_S$, kde $S$ je výchozí symbol gramatiky.
  
  \enlargethispage{1em}
   \begin{procedure}[H]
  \caption{Aktualizuj($P$, $B$)}
 \SetNlSty{}{}{:}
 \SetNlSkip{-1.0em}
 \SetInd{0.5em}{0.5em}
 \Indentp{1.4em}
    \ForEach{$A \rightarrow X_1X_2 \in P$, kde $X_1 = B \vee X_2 = B$, $X_1, X_2, A \in N'$}
    {
      \If{$M_A > M_{X_1} + M_{X_2}$}
      {
	$M_A := M_{X_1} + M_{X_2} $ \\
	Aktualizuj($P$, $A$)
      }
    }
\end{procedure}
  
  \begin{algorithm}[H]
 \SetKwInput{Input}{Vstup}\SetKwInOut{Output}{Výstup}
 \SetNlSty{}{}{:}
 \SetNlSkip{-1.0em}
 \SetInd{0.5em}{0.5em}
 \Input{Bezkontextová gramatika $G = (N,\Sigma, P, S)$}
 \Output{Váha bezkontextové gramatiky, $||G||$}
 \BlankLine
 \Indentp{1.4em}
    Převedení $G$ na ekvivalentní gramatiku $\overline{G}$ bez zbytečných symbolů.\\
    Převedení gramatiky $\overline{G}$ na ekvivalentní gramatiku $G'$ v Chomského normální formě, $G' = (N', \Sigma', P', S)$\\
    Polož $M_A := \infty$ pro každé $A \in N'$\\
    Polož $N_0 := \emptyset$, $P_0 = \emptyset$, $i := 1$\\
    \Repeat{$N_{i - 1} = N_{i-2}$}
    {
      $N_i := \{A| A\rightarrow \alpha \in P' \wedge \alpha\in(N_{i-1} \cup \Sigma)^*\} \cup N_{i-1}$\\
      $P_i := \{A \rightarrow \alpha | A \rightarrow \alpha \in P' \wedge \alpha\in(N_{i-1} \cup \Sigma)^*\}$\\
      \ForEach{$A \rightarrow \alpha \in P_i$}
      {
	\uIf{ $\alpha = x$, kde $x\in\Sigma'\ \mathbf{and}\ M_A > ||x||$}
	{
	  $M_A := ||x||$\\
	  Aktualizuj($P_i$, $A$)
	}
	\ElseIf{$\alpha = CD$, kde $C, D \in N'\ \mathbf{and}\ M_A > M_C + M_D$}
	{
	  $M_A := M_C + M_D$\\
	  Aktualizuj($P_i$, $A$)
	}
      }
      $i := i + 1$\\
    }
    \Return $M_S$
 \caption{\textsc{Minimální váha gramatiky, $\Sigma \subset \mathbb{N}$}}
 \label{fastSlam}
\end{algorithm}
\enlargethispage{1em}
  \item Pokud $\Sigma \subset\mathbb{Z}$, může nastat situace, kdy $M_A = -\infty$.
  Toto může nastat tehdy, pokud v $G$ existuje derivace $A\Rightarrow^*\alpha A \beta$ a 
  $M_\alpha + M_\beta < 0$, kde $\alpha, \beta \in (N \cup \Sigma)^*$ a $M_\alpha$ je součet minimálních vah 
  symbolů v řetězci $\alpha$. Pro detekci těchto záporných derivací upravíme praceduru Aktualizuj.
  Tuto proceduru rozšíříme o možnost uložení (množina $T$) všech pravidel, které vedly ke snížení hodnoty $M_A$.
  Pokud použijeme pravidlo, které se již v množině $T$ vyskytuje a pokud navíc dojde opět ke snížení hodnoty $M_A$, 
  znamená to, že $A \Rightarrow^*\alpha A \beta$ a navíc došlo ke snížení hodnoty $M_A$, tedy $M_A > M_A + M_\alpha + M_\beta$.
  Což znamená, že $M_\alpha + M_\beta < 0$ a vznikla tedy záporná rekurze a tom případě nastavíme $M_A := -\infty$.
  
  \enlargethispage{1em}
   \begin{procedure}[H]
  \caption{Aktualizuj($P$, $B$, $T$)}
 \SetNlSty{}{}{:}
 \SetNlSkip{-1.0em}
 \SetInd{0.5em}{0.5em}
 \Indentp{1.4em}
    \ForEach{$A \rightarrow X_1X_2 \in P$, kde $X_1 = B \vee X_2 = B$, $X_1, X_2, A \in N'$}
    {
      \uIf{$M_A > M_{X_1} + M_{X_2}$}
      {
	\uIf{$A \rightarrow X_1X_2 \in T$}
	{
	  $M_A := -\infty$
	}
	\Else
	{
	  $M_A := M_{X_1} + M_{X_2} $
	}
	Aktualizuj($P$, $A$, $T\cup \{A \rightarrow X_1X_2\}$)
      }
    }
\end{procedure}

    \begin{algorithm}[H]
 \SetKwInput{Input}{Vstup}\SetKwInOut{Output}{Výstup}
 \SetNlSty{}{}{:}
 \SetNlSkip{-1.0em}
 \SetInd{0.5em}{0.5em}
 \Input{Bezkontextová gramatika $G = (N,\Sigma, P, S)$}
 \Output{Váha bezkontextové gramatiky, $||G||$}
 \BlankLine
 \Indentp{1.4em}
    Převedení $G$ na ekvivalentní gramatiku $\overline{G}$ bez zbytečných symbolů.\\
    Převedení gramatiky $\overline{G}$ na ekvivalentní gramatiku $G'$ v Chomského normální formě, $G' = (N', \Sigma', P', S)$\\
    Polož $M_A := \infty$ pro každé $A \in N'$\\
    Polož $N_0 := \emptyset$, $P_0 = \emptyset$, $i := 1$\\
    \Repeat{$N_{i - 1} = N_{i-2}$}
    {
      $N_i := \{A| A\rightarrow \alpha \in P' \wedge \alpha\in(N_{i-1} \cup \Sigma)^*\} \cup N_{i-1}$\\
      $P_i := \{A \rightarrow \alpha | A \rightarrow \alpha \in P' \wedge \alpha\in(N_{i-1} \cup \Sigma)^*\}$\\
      \ForEach{$A \rightarrow \alpha \in P_i$}
      {
	\uIf{ $\alpha = x$, kde $x\in\Sigma'\ \mathbf{and}\ M_A > ||x||$}
	{
	  $M_A := ||x||$\\
	  Aktualizuj($P_i$, $A$, $\{A \rightarrow \alpha\}$)
	}
	\ElseIf{$\alpha = CD$, kde $C, D \in N'\ \mathbf{and}\ M_A > M_C + M_D$}
	{
	  $M_A := M_C + M_D$\\
	  Aktualizuj($P_i$, $A$, $\{A \rightarrow \alpha\}$)
	}
      }
      $i := i + 1$\\
    }
    \Return $M_S$
 \caption{\textsc{Minimální váha gramatiky, $\Sigma \subset \mathbb{Z}$}}
 \end{algorithm}
 \end{enumerate}

 
 \item {\bfseries Příklad}
    \begin{itemize}
     \item[--] Tvrzení dokážeme indukcí k počtu symbolů 0 ve slově $w \in L$ (počet symbolů 0 je označen jako $n$) .
     \item[--] {\it Bázový případ} $n = 0$ (počet symbolů 0 je roven nule). V tomto případě $w = 0^01^k \in L$, kde
      $0 \leq k \leq 0$. Tedy $k = 0$ a $w = 0^01^0 = \varepsilon \in L$. V gramatice $G$ ale existuje
      derivace $S \Rightarrow \varepsilon$ a tedy $w = \varepsilon \in L(G)$.
     \item[--] {\it Indukční krok}. Předpokládejme, že tvrzení platí pro všechny řetězce $w' \in L$, kde $\#_0(w')\leq n$
      pro nějaké $n\geq 0$. Ukážeme, že tvrzení platí i pro řetězce $w$ takové, že $\#_0(w) = n + 1$.
      \begin{itemize}
       \item Nejprve zvolme řetězec $w \in L$ takový, že platí $\#_0(w) = n + 1 > 0$. Tedy $w = 0^{n+1}1^k$, kde
      $0\leq 2(n+1) \leq k \leq 3(n+1)$. 
      
      \item Podle indukčního předpokladu, pro řetězec $w'= 0^n1^l \in L$, kde $0\leq 2n\leq l \leq 3n$
      a $\#_0(w') = n$, platí, že $w'\in L(G)$. Tím pádem existuje derivace $S \Rightarrow^* w'$. 
      
      \item Vzhledem k tomu, že $0 \leq 2n + 2 \leq k$ a $0 \leq 2n \leq l$, pro $n > 0$, tak $2 \leq k - l$.
      Podobně z druhé části nerovnice $k \leq 3n + 3$ a $l \leq 3n$ dostáváme $k - l \leq 3$. Dohromady tedy máme
      $2 \leq k-l \leq 3$. V řetězci $w$ je 
      tedy o 2 nebo o 3 více symbolů 1 než v řetězci $w' \in L(G)$. Řetězec $w$ tedy můžeme zapsat následovně:
      $w = 0w'11 = 00^n1^l11$ nebo $w = 0w'111 = 00^n1^l111$. Nyní budeme postupně zkoumat oba případy.
      \begin{enumerate}
       \item Nejprve případ, kdy $w = 00^n1^l11$. V gramatice $G$ máme pravidlo $S\rightarrow 0S11$. S využitím výše uvedeného
        dostáváme $S \Rightarrow 0S11 \Rightarrow^* 00^n1^l11 = w$ a tedy $S \Rightarrow^* w$ a tudíž $w \in L(G)$.
        
       \item Nyní případ, kdy $w = 00^n1^l111$. V gramatice $G$ máme pravidlo $S\rightarrow 0S111$. Opět s využitím výše uvedeného
        dostáváme $S \Rightarrow 0S111 \Rightarrow^* 00^n1^l111 = w$ a tedy $S \Rightarrow^* w$ a tudíž opět $w \in L(G)$.
      \end{enumerate}
      \end{itemize}
      \item [--] Dokázali jsme tedy, že $L\subseteq L(G)$.
    \end{itemize} 
    
 \item {\bfseries Příklad}
 \begin{enumerate}
  \item Gramatika není jednoznačná, je víceznačná, protože generuje víceznačnou větu 
  $w = \mathbf{if}\ cond\ \mathbf{if}\ cond\ com\ \mathbf{else}\ com$. Této větě odpovídají
  dva derivační stromy:
  \begin{center}
   \begin{forest}
      [$S$ [$\mathbf{if}$] [$cond$] [$S$ [$\mathbf{if}$] [$cond$] [$S$ [$com$]] [$\mathbf{else}$] [$S$ [$com$]]]]
    \end{forest}
    \begin{forest}
      [$S$ [$\mathbf{if}$] [$cond$] [$S$ [$\mathbf{if}$] [$cond$] [$S$ [$com$]]] [$\mathbf{else}$] [$S$ [$com$]]]
    \end{forest}
   \end{center}
   
   \item Jazyk $L(G)$ není jazyk s inherentní víceznačností, protože pro něj existuje jednoznačná bezkontextová gramatika $G'$.
   Tato gramatika je definována jako $G' = (\{S, B\}, \{\mathbf{if}, \mathbf{else}, cond, com\}, P, S)$ s pravidly
   \begin{eqnarray*}
      S &\rightarrow& \mathbf{if}\ cond\ S\ |\ \mathbf{if}\ cond\ B\ \mathbf{else}\ S\ |\ com\\
      B &\rightarrow& \mathbf{if}\ cond\ B\ \mathbf{else}\ B\ |\ com
   \end{eqnarray*}

   \item Výsledný deterministický zásobníkový automat přijímající jazyk $L(G)$:
   \begin{center}
      \begin{tikzpicture}[->,>=stealth',shorten >=1pt,auto,node distance=3.4cm,
			  semithick]
	\tikzstyle{every state}=[text=black]

	\node[state,initial,initial text={\#}] (A)             {$q_0$};
	\node[state]         (B) [right of=A,above of=A] {$q_1$};
	\node[state,accepting]         (C) [right of=A,below of=A] {$q_2$};
	\node[state,accepting]         (D) [node distance=9cm,right of=A] {$q_3$};

	\path (A) edge  [align=left,bend left]            node {$\mathbf{if},\#/0\#$ \\ $\mathbf{if},0/00$} (B)
	      (B) edge [align=left,bend left] node {$cond,0/0$} (A)
	      (A) edge  [align=left]            node {$com,\#/\#$} (D)
	      (A) edge  [align=left,bend left]            node {$com, 0/0$} (C)
	      (C) edge  [align=left,bend left]            node {$\mathbf{else},0/\varepsilon$} (A);
      \end{tikzpicture}
      \end{center}
     Přijetí slova $w = \mathbf{if}\ cond\ \mathbf{if}\ cond\ com\ \mathbf{else}\ com\ \mathbf{else}\ \mathbf{if}\ cond\ com$:
     \begin{eqnarray*}
      (q_0, w, \#) &\vdash& (q_1, cond\ \mathbf{if}\ cond\ com\ \mathbf{else}\ com\ \mathbf{else}\ \mathbf{if}\ cond\ com, 0\#) \vdash \\
      &\vdash& (q_0, \mathbf{if}\ cond\ com\ \mathbf{else}\ com\ \mathbf{else}\ \mathbf{if}\ cond\ com, 0\#) \vdash\\
      &\vdash& (q_1, cond\ com\ \mathbf{else}\ com\ \mathbf{else}\ \mathbf{if}\ cond\ com, 00\#) \vdash\\ 
      &\vdash&(q_0, com\ \mathbf{else}\ com\ \mathbf{else}\ \mathbf{if}\ cond\ com, 00\#) \vdash\\
      &\vdash& (q_2, \mathbf{else}\ com\ \mathbf{else}\ \mathbf{if}\ cond\ com, 00\#) \vdash \\
      &\vdash&(q_0, com\ \mathbf{else}\ \mathbf{if}\ cond\ com, 0\#) \vdash (q_2, \mathbf{else}\ \mathbf{if}\ cond\ com, 0\#) \vdash\\
      &\vdash&(q_0, \mathbf{if}\ cond\ com, \#) \vdash (q_1, cond\ com, 0\#) \vdash (q_0, com, 0\#) \vdash\\
      &\vdash&(q_2, \varepsilon, 0\#) 
     \end{eqnarray*}
 \end{enumerate}


\end{enumerate}



\end{document}