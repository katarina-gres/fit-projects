\documentclass[a4paper,12pt]{article}

\usepackage[utf8]{inputenc}
\usepackage[czech]{babel}
\usepackage[IL2]{fontenc}
\usepackage[left=3cm,text={15cm, 23cm},top=3.5cm]{geometry}

\usepackage{graphicx}
\usepackage{tikz}

\usepackage{amssymb}
\usepackage{amsmath}
\usepackage{amsthm}
\usepackage{forest}
\usepackage[ruled,czech,linesnumbered,noline]{algorithm2e}
\usepackage{array}
\usepackage{multirow}
\usepackage{textcomp}

\usepackage{xpatch}
\xpretocmd{\algorithm}{\hsize=\linewidth}{}{}
\xpretocmd{\procedure}{\hsize=\linewidth}{}{}

\usepackage{enumitem}
%\usepackage[neverdecrease]{paralist}

\usepackage{pgf}
\usepackage{tikz}
\usetikzlibrary{arrows,automata}

\newcounter{counten}
\setcounter{counten}{1}

\newtheorem{theorem}{Tvrzení}[counten]
\newtheorem{corollary}{Důsledek}[counten]



\title{Úkol č. 3 do předmětu Teoretická informatika}
\author{Vojtěch Havlena (xhavle03)}
\date{}

\begin{document}

\maketitle

\setlist[enumerate,1]{leftmargin=0.2cm}

\begin{enumerate}[label=\textbf{\arabic*}.]

 \item {\bfseries Příklad}
 
 Jazyk Turingova stroje na obrázku lze pomocí regulárního výrazu zapsat jako $a(baa)^*$. Gramatika
 $G$ generující tento jazyk je potom definována následovně $G = (\{S, B\}, \{a, b\}, P, S)$, kde
 množina přepisovacích pravidel je dána jako $$P = \{S \rightarrow aB, B\rightarrow baaB, B\rightarrow\varepsilon\}.$$
 
 Přijímající běh Turingova stroje pro řetězec $abaa$.
 \begin{eqnarray*}
   (0, \Delta abaa\Delta^{\omega}, 0) &\vdash& (1, \Delta abaa\Delta^{\omega}, 1) \vdash (2, \Delta abaa\Delta^{\omega}, 2) 
   \vdash (3, \Delta abaa\Delta^{\omega}, 3) \\
   &\vdash& (1, \Delta abaa\Delta^{\omega}, 4) \vdash (2, \Delta abaa\Delta^{\omega}, 5) 
   \vdash (5, \Delta abaa\Delta^{\omega}, 4) \\ &\vdash& (6, \Delta aba\Delta^{\omega}, 4) \vdash (5, \Delta aba\Delta^{\omega}, 3)
   \vdash (6, \Delta ab\Delta^{\omega}, 3) \\ &\vdash& (5, \Delta ab\Delta^{\omega}, 2) \vdash (6, \Delta a\Delta^{\omega}, 2)
   \vdash (5, \Delta a\Delta^{\omega}, 1) \\ &\vdash& (6, \Delta^{\omega}, 1) 
   \vdash (5, \Delta^{\omega}, 0) \vdash (7, \Delta^{\omega}, 1) \\
   &\vdash& (8, \Delta Y\Delta^{\omega}, 1) \vdash (9, \Delta Y\Delta^{\omega}, 0) 
 \end{eqnarray*}
 
 Derivace řetězce $abaa$ gramatikou $G$.
 $$
  S \Rightarrow aB \Rightarrow abaaB \Rightarrow abaa
 $$


\item {\bfseries Příklad}
 
 Důkaz toho, že problém prázdnosti jazyka daného Turingova stroje (EMP) není ani částečně rozhodnutelný provedu
 redukcí z co-HP, kde $$\mbox{EMP} = \{\langle M\rangle\ | L(M) = \emptyset\} \mbox{ a }\mbox{co-HP }= \{\langle M\rangle\#\langle w\rangle | M \mbox{ nezastaví při } w\}.$$ Jazyk co-HP není ani
 rekurzivně vyčíslitelný jazyk. Navrhneme tedy totální funkci $\sigma: \{0,1,\#\}^* \rightarrow \{0,1\}^*$, která zachovává
 členství v jazycích co-HP a EMP a je implementovatelná úplným TS. Funkce $\sigma$ pro zadaný vstup $x\in\{0,1\}^*$
 vygeneruje kód TS $M_x$ (vygeneruje tedy $\langle M_x\rangle$) a $M_x$ pracuje následovně:
 \begin{enumerate}
  \item $M_x$ smaže obsah své vstupní pásky.
  \item $M_x$ zapíše na svou pásku řetězec $x$.
  \item $M_x$ otestuje, zda řetězec $x$ má strukturu $x_1\#x_2$, kde $x_1 = \langle M\rangle$ ($x_1$ je tedy kód TS $M$) a 
    $x_2$ je kód jeho vstupu $w$. Pokud takovou strukturu nemá, přijme.
  \item Jinak $M_x$ s využitím univerzálního TS odsimuluje běh TS $M$ na vstupu $w$. Pokud $M$ na $w$ zastaví, 
    $M_x$ přijme řetězec $x$ (jinak tedy cyklí).
 \end{enumerate}
 Redukci $\sigma$ lze implementovat úplným TS. Stačí, aby tento úplný TS byl schopen vygenerovat TS, který
 předává řízení mezi 4 komponentami, které implementují body (a) -- (d). Komponenty (a), (b), (d) jsou 
 nezávislé na $x$. Pro vypsání $x$ na vstupní pásku stačí vygenerovat kód TS, který postupně zapisuje
 symboly z $x$ na vstupní pásku a posunuje hlavu o jednu pozici doprava.
 
 Jazyk $L(M_x)$ je buď prázdný nebo $\{0,1\}^*$, přičemž
 \begin{enumerate}
  \item $L(M_x) = \emptyset \Leftrightarrow x = x_1\#x_2$, kde $x_1 = \langle M\rangle$ ($x_1$ je kód TS $M$) a 
  $x_2 = \langle w\rangle$ (kód jeho vstupu) a $M$ na $w$ nezastaví.
  \item $L(M_x) = \{0,1\}^* \Leftrightarrow x$ není správně zformovaná instance co-HP nebo 
  $x = x_1\#x_2$, kde $x_1 = \langle M\rangle$ ($x_1$ je kód TS $M$) a 
  $x_2 = \langle w\rangle$ (kód jeho vstupu) a $M$ na $w$ zastaví.
 \end{enumerate}
 
 V posledním kroku důkazu zbývá ukázat, že $\sigma$ zachovává členství v jazyce, tj. $x\in\mbox{ co-HP} \Leftrightarrow \sigma(x)\in\mbox{ EMP}$.
 Tedy 
 
 $ \sigma(x) = \langle M_x\rangle \in\mbox{EMP} \Leftrightarrow L(M_x) = \emptyset \Leftrightarrow x = x_1\#x_2$, kde $x_1 = \langle M\rangle$ ($x_1$ je kód TS  $M$) a 
  $x_2 = \langle w\rangle$ (kód jeho vstupu) a $M$ na $w$ nezastaví $\Leftrightarrow x\in$ co-HP. Problém prázdnosti jazyka daného Turingova stroje tedy není
  ani částečně rozhodnutelný.

 \item {\bfseries Příklad}

 \begin{enumerate}
  \item Předpokládejme, že existuje konečná množina $P$ řetězců, pro kterou je $HP$ $P$-rozhodnutelný. Tedy existuje TS $M_P$, který $P$-rozhoduje $HP$. Potom 
    je jazyk $PHP = HP\setminus P$ rekurzivní -- úplný TS $M$, který rozhoduje jazyk $PHP$ by pracoval následovně: TS $M$ by zkontroloval, zda vstupní řetězec je
    obsažen v množině $P$ (množina $P$ je konečná, stačí projít všechny prvky a porovnávat). Pokud ano, $M$ zamítne. Jinak pomocí univerzálního TS
    simuluje běh TS $M_P$, který $P$-rozhoduje $HP$ se vstupním řetězcem. Pokud $M_P$ přijme, tak přijme i $M$. Pokud $M_P$ zamítne, zamítne i $M$.
    
    Dále je možné si jazyk $HP$ vyjádřit následovně: $HP = PHP \cup (HP \cap P)$. Vzhledem k tomu, že $P$ je konečná množina, je i
    $HP \cap P$ konečná množina a tedy i rekurzivní jazyk. Vzhledem k tomu, že $PHP$ je rekurzivní jazyk a rekurzivní jazyky jsou uzavřeny
    vůči sjednocení, je i $HP = PHP \cup (HP \cap P)$ rekurzivní jazyk, což je spor. Konečná množina $P$ řetězců, pro kterou je
    $HP$ $P$-rozhodnutelný tedy neexistuje.
    
  \item Za $P$ si zvolím celou množinu $HP$, tedy $P = HP$. Tato množina je nekonečná. Dále uvažujme TS $M$ takový, který pro každý
   vstup okamžitě zastaví a nepřijme. Tedy $L(M) = \emptyset$. Tento TS pro každý vstup zastaví a ze slov mimo $P$ přijímá právě ta, která patří do
   $HP$ (což nejsou žádná slova). A tedy TS $M$ $P$-rozhoduje $HP$.
   
  \item Předpokládejme, že tvrzení platí, tedy pro všechny nekonečné množiny řetězců $P$ je $HP$ $P$-rozhodnutelný. Když platí pro všechny nekonečné,
  tak platí i pro $P = \{0,1\}^*$. 
  Nechť pro $x\in\{0,1\}^*$ je $M_x$ TS s kódem $x$, je-li $x$ validní kód TS. Jinak $M_x$ je pevně zvolený TS, který pro libovolný vstup okamžitě zastaví.
  Posloupnost $M_\varepsilon, M_0, M_1, \dots$, idexovaná řetězci z $\{0,1\}^*$ zahrnuje všechny Turingovy stroje nad $\{0,1\}^*$.
  Podle předpokladu existuje TS $K$, který $P$-rozhoduje $HP$. Tento TS pro vstup $\langle M\rangle\#\langle w\rangle$ (platí, že  $\langle M\rangle\#\langle w\rangle\notin P$):
  \begin{itemize}
   \item[--] Zastaví a přijme právě tehdy, když TS $M$ zastaví na $w$.
   \item[--] Zastaví a odmítne právě tehdy, když TS $M$ cyklí na $w$.
  \end{itemize}
  Dále je možné sestavit TS $N$, který pro vstup $x\in\{0,1\}^*$:
  \begin{itemize}
   \item[--] Sestaví $M_x$ z $x$ a zapíše $\langle M_x\rangle\#\langle x\rangle$ na svou pásku.
   \item[--] Dále simuluje TS $K$ na vstup $\langle M_x\rangle\#\langle x\rangle$ a přijme, pokud $K$ odmítne. Pokud
   $K$ přijme, přejde do nekonečného cyklu.
  \end{itemize}
  Potom ale $N$ zastaví na $x$ $\Leftrightarrow$ $K$ odmítne $\langle M_x\rangle\#\langle x\rangle \Leftrightarrow M_x$ cyklí na $x$.
  TS $N$ se tedy liší od každého $M_x$ alespoň na jednom řetězci. Což je ovšem spor s tím, že posloupnost $M_\varepsilon, M_0, M_1, \dots$
  zahrnuje všechny Turingovy stroje nad $\{0,1\}$. Tvrzení tedy neplatí.

 \end{enumerate}

 
 \item {\bfseries Příklad}
 Pro simulaci Turingova stroje (TS) pomocí rendez-vous sítí (RVS) je nejprve nutné uvažovat kódování 
 konfigurace TS pomocí rendez-vous sítí. Neformálně každý proces v RVS představuje políčko na pásce TS.
 Tyto procesy jsou lineárně zřetězeny pomocí komunikačních kanálů. Každý proces (vyjma procesu představující 1. 
 políčko pásky a poslední neblankové políčko pásky) je pomocí jednoho kom. kanálu spojen s jedním sousedem a 
 pomocí druhého kom. kanálu s druhým sousedem (lze si představit jako jakýsi lineárně vázaný seznam).
 
 Ve stavech automatu popisující chování procesů ukládám informaci o aktuálním stavu TS, páskovém 
 symbolu, který je uložen na políčku pásky reprezentovaném procesem, číslo procesu pro oddělení jednotlivých procesů a příznak, který 
 udává, zda tento proces má řízení (= hlava TS je právě nad tímto políčkem pásky).
 Příznakem řízení může být označen pouze jeden proces
 Pro číselné oddělení jednotlivých procesů používám čísla ze $\mathbb{Z}_3$, navíc s číslem $4$ pro 
 číslo nově přidaného procesu (=posunutí hlavy TS na pozici pásky, pro kterou zatím neexistuje proces). Stav konečného
 automatu je tedy čtveřice.
 
 Abeceda je trojice stav TS (pro předávání řízení mezi procesy), identifikace procesu a příznak popisující, zda
 se jedná a nově vytvořený proces.

 Pro daný TS $M = (Q_{T}, \Sigma_{T}, \Gamma, \delta_T, q^T_0, q^T_F)$ je konečný automat $A$ popisující chování procesů
 definován následovně: $A = (Q, \Sigma\times\{0,1\}, \delta, q_0, F)$, kde 
 \begin{itemize}
  \item[--] $Q = Q_{T}\times\Gamma\times M\times\{0,1\}$, kde $M = \mathbb{Z}_3\cup \{4\}$
  \item[--] $\Sigma = Q_T\times M\times\{0,1\}$
  \item[--] $q_0 = (q^T_0, \Delta, 4, 0)$
  \item[--] $F = \{q^T_F\}\times\Gamma\times M\times\{0,1\}$
  \item[--] Přechodová funkce $\delta: Q\times\Sigma\times\{0,1\}\rightarrow 2^Q$ je dána následovně ($\oplus$ značí operaci sčítání v $\mathbb{Z}_3$
   a $\ominus$ odečítání v $\mathbb{Z}_3$):
  \begin{eqnarray*}-
    &&\forall q_1, q_2, q_3 \in Q_T, \forall a_1, a_3\in\Gamma,\forall v_1, v_2, v_3\in M, \forall Y_1, Y_3, new, i \in \{0,1\}: \\
    &&(q_3, a_3, v_3, Y_3) \in \delta((q_1, a_1, v_1, Y_1), ((q_2, v_2, new), i)) \Leftrightarrow \\ 
    && (new = 0  \wedge Y_1 = 0 \wedge Y_3 = 1 \wedge v_1 = v_2 = v_3 \neq 4 \wedge a_3 = a_1 \wedge q_3 = q_2) \vee \\
    && (new = 0  \wedge Y_1 = 1 \wedge Y_3 = 0 \wedge v_2 = v_1 \oplus 1 \wedge v_3 = v_1 \wedge a_3 = a_1\wedge \\ 
    &&\quad\quad\quad\quad\quad q_3 = q_2 \wedge \delta_T(q_1, a_1) = (q_2, R)) \vee \\
    && (new = 0  \wedge Y_1 = 1 \wedge Y_3 = 0 \wedge v_2 = v_1 \ominus 1 \wedge v_3 = v_1 \wedge a_3 = a_1\wedge \\ 
    &&\quad\quad\quad\quad\quad q_3 = q_2 \wedge \delta_T(q_1, a_1) = (q_2, L)) \vee \\
    &&(new = 0  \wedge Y_1 = 0 \wedge Y_3 = 0 \wedge v_2 = 4 \wedge v_1 = v_3 \neq 4 \wedge a_3 = a_1 \wedge q_1 = q_3) \vee \\
    &&(new = 0  \wedge Y_1 = 1 \wedge Y_3 = 1 \wedge v_2 = 4 \wedge v_1 = v_3 \neq 4 \wedge \delta_T(q_1, a_1) = (q_3, a_3)) \vee \\
    && (new = 1  \wedge Y_1 = 1 \wedge Y_3 = 0 \wedge v_2 = v_1 = v_3 \wedge a_3 = a_1\wedge \\
      &&\quad\quad\quad\quad\quad q_3 = q_2 \wedge \delta_T(q_1, a_1) = (q_2, R)) \vee \\
    && (new = 1  \wedge Y_1 = 0 \wedge Y_3 = 1 \wedge v_1 = 4 \wedge v_3 = v_2 \oplus 1 \wedge a_3 = a_1 \wedge q_3 = q_2)
  \end{eqnarray*}
  Sémantika: $q$ je stav TS, $a$ je symbol na pásce, $v$ je identifikace procesu, $Y$ je příznak řízení, $new$ -- zda se jedná o nový proces. 
  První tři pravidla představují předání řízení z aktivního procesu do sousedního procesu (simulace posunu hlavy TS vlevo/vpravo). Další dvě pravidla
  představují simulaci pravidla, kdy v TS dojde k přepsání symbolu pod hlavou (samotná hlava není posunuta). Poslední dvě pravidla
  simulují posun hlavy vpravo za poslední neblankový symbol na pásce (dojde k vytvoření nového procesu, tomuto procesu se přiřadí identifikační číslo
  a předá se mu řízení).
 \end{itemize}
 
 \textbf{Způsob kódování konfigurace TS pomocí RVS.} Nechť $(q, \gamma\Delta^{\omega}, n)$ je konfigurace TS $M$, kde $\gamma\in\Gamma^*$ a $|\gamma| > n$. Potom této konfiguraci odpovídá konfigurace RVS $(S, stav)$,
 kde $S = (A, P, K)$. Konečný automat $A$ je pro TS $M$ definován, jak je popsáno výše. Množina procesů $P = \{p_0, \dots, p_{m-1}\}$, kde $m = |\gamma|$.
 Množina komunikačních kanálů $K = \{(p_i, 1, p_{i+1})| p_i, p_{i+1}\in P, i \mbox{ je sudé}\}\cup \{(p_i, 2, p_{i+1})| p_i, p_{i+1}\in P, i\mbox{ je liché}\}$.
 Funkce $stav$ je potom definována následovně:
 $$
  stav(p_i) = \begin{cases} (q^T_0, \gamma_i, i\mod 3, 0)\quad\mbox{pokud } i\neq n\\ (q, \gamma_i, i\mod 3, 1)\quad\mbox{pokud } i= n \end{cases}
 $$
 V případě, kdy máme zadanou konfiguraci RVS, tak odpovídající konfiguraci TS získáme tam, že nelezneme proces, který má nastaven příznak řízení. 
 Z toho procesu získáme aktuální stav TS. Pozici hlavy získáme průchodem z aktivního procesu přes komunikační kanály směrem k procesům s nižším identifikačním číslem.
 
 \begin{theorem}
 Nechť $M$ je Turingův stroj a $K_1 = (q, \gamma\Delta^{\omega}, n)$ a $K_2 = (q', \gamma'\Delta^{\omega}, m)$ jsou konfigurace $M$, přičemž $K_1\neq K_2$. Dále předpokládejme, že
 $(S, stav)$ je konfigurace RVS odpovídající konfiguraci $K_1$ a $(S', stav')$ je konfigurace RVS 
 odpovídající konfiguraci $K_2$ a navíc $(S, stav) \neq (S', stav')$. Potom $K_1 \vdash_M K_2$
 právě tehdy když $(S, stav) \rightarrow (S', stav')$.
 \end{theorem}
 \begin{proof}
 $(\Rightarrow)$ Mohou nastat následující 3 možnosti:
 \begin{enumerate}
  \item $m = n$ a $\gamma' = s_b^n(\gamma)$ a $\delta_T(q, \gamma_n) = (q', b)$.
  
   V tomto případě $p_n\in P$ je proces, který má řízení a tedy $stav(p_n) = (q, \gamma_n, v_n, 1)$.
   Nehchť $p_m$ je libovolný proces, který je s $p_n$ spojen komunikačním kanálem $i\in\{0,1\}$ a tedy $stav(p_m) = (q'', a, v_m, 0)$,
   kde $v_m = v_n \oplus 1$ nebo $v_m = v_n \ominus 1$. Potom existuje $(q, 4, 0) \in\Sigma$ a 
   \begin{eqnarray*}
      stav'(p_n) &=& \delta((q, \gamma_n, v_n, 1), ((q, 4, 0),i)) = (q',s_b^n(\gamma), v_n, 1)\\
      stav'(p_m) &=& \delta((q'', a, v_m, 0),((q, 4, 0),i)) = (q'', a, v_m, 0).
   \end{eqnarray*}
   Tedy $(S, stav) \rightarrow (S, stav')$ a $(S, stav')$ je konfigurace RVS, která odpovídá konfiguraci TS $M$ $(q', \gamma'\Delta^{\omega}, m)$.
   
  \item $m = n + 1$ a $\gamma' = \gamma$ a $\delta_T(q, \gamma_n) = (q', R)$.
  
  Opět $p_n\in P$ je proces, který má řízení a tedy $stav(p_n) = (q, \gamma_n, v_n, 1)$.
  Nejprve předpokládejme, že $p_n$ je spojen oběma komunikačními kanály. Potom $p_m$ je libovolný proces, který je s $p_n$ spojen komunikačním kanálem $i\in\{0,1\}$ a jeho identifikátor
  $v_m = v_n \oplus 1$ a tedy $stav(p_m) = (q'', a, v_n\oplus1, 0)$. Potom existuje $(q', v_n\oplus1, 0) \in\Sigma$ a 
   \begin{eqnarray*}
      stav'(p_n) &=& (q',\gamma_n, v_n, 0) = \delta((q, \gamma_n, v_n, 1), ((q', v_n \oplus 1, 0),i))\\
      stav'(p_m) &=& (q', a, v_n\oplus1, 1) = \delta((q'', a, v_n \oplus 1, 0),((q', v_n\oplus1, 0),i)).
   \end{eqnarray*}
   Tedy $(S, stav) \rightarrow (S, stav')$ a $(S, stav')$ je konfigurace RVS, která odpovídá konfiguraci TS $M$ $(q', \gamma'\Delta^{\omega}, m)$.
  Nyní uvažujme případ, kdy $p_n$ není spojen s procesem, jehož identifikátor je $v_n \oplus 1$ (musíme tedy přidat nový proces $p_m$).
  \begin{eqnarray*}
      stav'(p_n) &=& (q',\gamma_n, v_n, 0) = \delta((q, \gamma_n, v_n, 1), ((q', v_n, 1),i))\\
      stav'(p_m) &=& (q', \Delta, v_n\oplus 1, 1) = \delta((q^T_0, \Delta, 4, 0),((q', v_n, 1),i)).
   \end{eqnarray*}
   Opět tedy Tedy $(S, stav) \rightarrow (S', stav')$ a $(S', stav')$ je konfigurace RVS, která odpovídá konfiguraci TS $M$ $(q', \gamma'\Delta^{\omega}, m)$
   
   \item $m = n - 1$ a $\gamma' = \gamma$ a $\delta_T(q, \gamma_n) = (q', L)$, $n > 0$.
  
  Opět $p_n\in P$ je proces, který má řízení a tedy $stav(p_n) = (q, \gamma_n, v_n, 1)$ a
  $p_m$ je proces, který je s $p_n$ spojen komunikačním kanálem $i\in\{0,1\}$ a pro jeho identifikátor platí
  $v_m = v_n \ominus 1$. Tedy $stav(p_m) = (q'', a, v_n\ominus1, 0)$. Potom existuje $(q', v_n\ominus1, 0)\in\Sigma$: 
   \begin{eqnarray*}
      stav'(p_n) &=& (q',\gamma_n, v_n, 0) = \delta((q, \gamma_n, v_n, 1), ((q', v_n \ominus 1, 0),i))\\
      stav'(p_m) &=& (q', a, v_n\ominus1, 1) = \delta((q'', a, v_n \ominus 1, 0),((q', v_n\ominus1, 0),i)).
   \end{eqnarray*}
   Tedy $(S, stav) \rightarrow (S, stav')$ a $(S, stav')$ je konfigurace RVS, která odpovídá konfiguraci TS $M$ $(q', \gamma'\Delta^{\omega}, m)$.
 \end{enumerate}
 
 $(\Leftarrow)$ Vzhledem k tomu, že $(S, stav) \neq (S', stav')$, tak spolu nemohly komunikovat žádné dva procesy, z nichž ani jeden nemá řízení.  
 Jediná možnost komunikace mezi procesy bez řízení je totiž 4. pravidlo přechodové funkce a toto pravidlo nemění stav.
 Komunikace tedy musela proběhnout mezi dvěma procesy z nichž právě jeden proces má řízení. Vzhledem k tomu, jak je
 přechodová funkce $\delta$ definována, to znamená, že ze stavu a aktuálního symbolu, který reprezentuje aktivní proces existuje
 přechod v TS $M$. Tedy $K_1 \vdash_M K_2$ a nový stav RVS odpovídá konfiguraci TS $K_2$.
 \end{proof}
 \begin{corollary}
   Nechť $M$ je TS, $(q, \gamma\Delta^{\omega}, n)$, kde $\gamma\in\Gamma^*$ je jeho konfigurace a $(S, stav)$ je konfigurace RVS odpovídající 
   konfiguraci $(q, \gamma\Delta^{\omega}, n)$. Potom $(q, \gamma\Delta^{\omega}, n) \\\vdash^*_M (q^T_F, \gamma'\Delta^{\omega}, m)$, kde $\gamma'\in\Gamma^*$ právě tehdy
   když z konfigurace $(S, stav)$ je dosažitelný koncový stav.
 \end{corollary}

 Dále předpokládejme, že existuje jednoznačné kódování stavu RVS pomocí symbolů z nějaké abecedy $\Sigma$. Kód konfigurace
 RVS $S$ potom označuji jako $\langle (S, stav)\rangle$. (Například $\Sigma = \{0,1,\#_1, \#_2\}$ a v kódu $S$ jsou
 jednotlivé složky odděleny symbolem $\#_1$, kódování automatu pomocí řetězce $\{0,1\}^*$ je podobné jako kódování TS.
 RVS od funkce $stav$ potom může být oddělen symbolem $\#_2$ a kódování funkce $stav$ může být provedeno tak, že jednotlivé procesy spolu s akt. stavy jsou odděleny symbolem $\#_1$).
 
 Nyní k samotnému důkazu, že problém dosažitelnosti koncového stavu (DOS) je nerozhodnutelný. Uvažujme funkci $\sigma: \{0,1,\#\}^*\rightarrow \Sigma^*$,
 která provádí redukci z problému náležitosti MP. $$\mbox{DOS} = \{\langle (S, stav)\rangle | \mbox{ Ze stavu } (S, stav) \mbox{ dosáhne S konc. konfigurace}\}$$
 Funkce $\sigma$ pro zadaný vstupní řetězec $x$ funguje následovně:
 \begin{itemize}
  \item[--] Pokud $x$ není správně zformovanou instancí MP, vrať kód RVS, která nikdy nedosáhne koncového stavu.
  \item[--] Pokud $x$ je ve tvaru $x_1\#x_2$, kde $x_1$ je kód TS $M$ a $x_2$ je kód jeho vstupu $w$, vygeneruj 
    kód RVS, která odpovídá počáteční konfiguraci TS $M$, tedy $(q_0^T, \Delta w\Delta^{\omega}, 0)$.
 \end{itemize}
 
 Podle důsledku 1.1. platí: $x\in $ MP $\Leftrightarrow \sigma(x) \in$ DOS. Tedy problém dosažitelnosti koncového
 stavu z dané konfigurace RVS je nerozhodnutelný.

\end{enumerate}



\end{document}