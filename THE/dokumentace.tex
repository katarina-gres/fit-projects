\documentclass[a4paper,12pt]{article}

\usepackage[utf8]{inputenc}
\usepackage[czech]{babel}
\usepackage[IL2]{fontenc}
\usepackage[left=3cm,text={15cm, 23cm},top=3.5cm]{geometry}
\usepackage{subcaption}
\usepackage{listings}
\usepackage{multirow}
\usepackage{array}
\renewcommand{\arraystretch}{2}

\usepackage{amsthm}

\usepackage{color}
\usepackage[ruled,czech,linesnumbered,noline]{algorithm2e}
\usepackage{graphicx}
\SetKwRepeat{Do}{do}{while}%

\usepackage{tikz}
\usetikzlibrary{patterns}

\definecolor{grey}{rgb}{0.745098,0.745098,0.745098}

\lstset{ %
  basicstyle=\scriptsize
}

\usepackage{amssymb}
\usepackage{amsmath}
\usepackage{amsthm}
\usepackage{tikz}

\newtheorem{theorem}{Věta}
\newtheorem{definition}{Definice}
\newtheorem{example}{Příklad}

\title{\bf Nashovo ekvilibrium s tolerancí}
\author{Vojtěch Havlena, xhavle03\\ \texttt{xhavle03@stud.fit.vutbr.cz}}
\date{}

\begin{document}
\maketitle

\begin{abstract}
  Vzhledem k výpočetní náročnosti hledání Nashových ekvilibrií v nekooperativních hrách, se nabízí otázka, 
  jestli je možné hledat pouze přibližná řešení (Nashova ekvilibria).
  Z tohoto důvodu byla zavedena Nashova ekvilibria s tolerancí (přibližná Nashova ekvilibria). Nashovo ekvilibrium s tolerancí je 
  strategický profil, který přibližně splňuje podmínky kladené na Nashovo ekvilibrium. V této práci jsou
  zkoumány dva přístupy k problematice Nashova ekvilibria s tolerancí\,--\,$\epsilon$-Nashovo ekvilibrium a
  okrajově také $\epsilon$-well-supported Nashovo ekvilibrium. V práci je také ukázáno, jaký vliv má použití Nashových 
  ekvilibrií s tolerancí na výpočetní složitost.
\end{abstract}

\section{Úvod}
Pravděpodobně nejdůležitější koncept nekooperativních her\footnote{V této práci uvažuji implicitně pouze konečné hry.} je Nashovo ekvilibrium. Jedná se o strategický profil, při kterém 
žádný hráč nemá důvod jednostranně změnit svoji strategii. Jinými slovy je to strategický profil, ve kterém žádný hráč 
nezvýší svůj zisk tím, že by zvolil jinou strategii, zatímco ostatní hráči by si zachovali svoji stávající strategii.

Ačkoliv již J. F. Nash dokázal, že každá konečná nekooperativní hra má řešení ve formě smíšeného Nashova ekvilibria\footnote{V dalším textu, pokud je řeč o Nashově ekvilibriu, myslím automaticky Nashovo ekvilibrium ve smíšených strategiích.}, 
samotné hledání Nashova ekvilibria je obecně velmi obtížný problém (existence polynomiálního algoritmu je 
otevřený problém). I~přesto existují algoritmy pro hledání Nashových ekvilibrií. Jedním
z nejznámějších je Lemke-Howsonův algoritmus pro hledání NE ve dvouhráčových hrách v normální formě (bimatrix games). Bylo ale ukázáno, že časová složitost tohoto
algoritmu je v nejhorším případě exponenciální vzhledem k počtu ryzích strategií hráčů~\cite{ECTA:ECTA667}. 

Možný způsob, jak se vypořádat se zmíněnou výpočetní složitostí, je hledat přibližná NE (tedy NE s určitou tolerancí). 
Jedním z konceptů přibližných Nashových ekvilibrií je $\epsilon$-Nashovo ekvilibrium.
Neformálně, smíšený profil je $\epsilon$-NE pokud žádný hráč nemůže zvýšit svůj očekávaný zisk o více než 
$\epsilon$ v případě, že by jednostranně změnil svoji strategii.

Formálněji, uvažujme konečnou nekooperativní hru $n$ hráčů v normální formě $\Gamma = (Q, \{S_i\}_{i\in Q}, \{U_i\}_{i\in Q})$ a její smíšené rozšíření
$\Gamma^m = (Q, \{\Delta_i\}_{i\in Q}, \{\pi_i\}_{i\in Q})$. Potom můžeme $\epsilon$-NE definovat následovně\footnote{Uvedená definice je pro 
smíšené $\epsilon$-Nashovo ekvilibrium, které je v této práci uvažováno implicitně. Velice podobným způsobem je možné 
definovat $\epsilon$-Nashovo ekvilibrium v ryzích strategiích.}.

\begin{definition}[\cite{Leyton-Brown:2008:EGT:1481632}]
  Mějme $\epsilon > 0$. Smíšený strategický profil $\sigma \in\Delta$ je $\epsilon$-Nashovo ekvilibrium ve hře $\Gamma$, pokud
  pro všechny $i\in Q$ a $\sigma_i'\in\Delta_i$ platí $\pi_i(\sigma) \geq \pi_i(\sigma_i', \sigma_{-i}) - \epsilon$.
\end{definition}

\emph{Pozn. V této práci se snažím co nejvíce využít notaci použitou na přednáškách (někdy se liší od notace, která se využívá v jednotlivých článcích).}
Pokud bychom v definici povolili $\epsilon = 0$, tak dostáváme definici Nashova ekvilibria. Každé Nashovo ekvilibrium je tedy i 
$\epsilon$-NE. Vzhledem k tomu, že pro každou konečnou nekooperativní hru
existuje Nashovo ekvilibrium, existuje i $\epsilon$-NE. Každé NE je aproximováno $\epsilon$-Nashovými ekvilibrii. 
Při výpočtu Nashových ekvilibrií na počítači je použití $\epsilon$-NE poměrně přirozené. Díky tomu, že reálná čísla jsou reprezentována
pouze s omezenou přesností, i algoritmy pro hledání Nashových ekvilibrií hledají ve skutečnosti $\epsilon$-Nashova 
ekvilibria pro hodnotu $\epsilon$, které odpovídá přesnosti práci s reálnými čísly~\cite{Leyton-Brown:2008:EGT:1481632}.

\begin{figure}[h]
  \catcode`\-=12
  \centering
  \begin{tabular}{ c | c | c | }
  \multicolumn{1}{c}{} & \multicolumn{1}{c}{$L$} & \multicolumn{1}{c}{$R$} \\\cline{2-3}
  $U$ & 1,1 & 0,0 \\\cline{2-3}
  $D$ & 1+$\frac{\epsilon}{2}$, 1 & 500,500 \\\cline{2-3}
\end{tabular}
\caption{Hra v příkladu 1.}
\label{fig:game}
\end{figure}


\begin{example}[\cite{Leyton-Brown:2008:EGT:1481632}]\label{ex:1}
  Uvažujme hru na obr.~\ref{fig:game} pro $\epsilon > 0$. Zde vidíme, že pro řádkového hráče jeho strategie $D$ striktně dominuje nad jeho strategií $U$. Po odstranění
  strikně dominované strategie $U$, strategie $R$ sloupcového hráče strikně dominuje nad jeho strategií $L$. Iterativním odstraňováním
  strikně dominovaných strategií jsme tedy zjistili, že hra má jediné Nashovo ekvilibrium $(D,R)$. Tento profil je rovněž
  $\epsilon$-NE. Nicméně v této hře je také profil $(U, L)$ $\epsilon$-NE, protože žádný hráč si změnou strategie nepolepší o více než 
  $\epsilon$ (řádkový hráč by si polepšil jen o hodnotu $\epsilon/2$).
\end{example}

I když každé Nashovo ekvilibrium je \uv{obklopeno} $\epsilon$-Nashovými ekvilibrii, tak ne každé $\epsilon$-Nashovo 
ekvilibrium je, co se týče očekávaného užitku, \uv{blízko} očekávaného užitku NE (viz profil $(U, L)$ ve hře z příkladu~\ref{ex:1}).

Na závěr ještě uvedeme další z konceptů přibližných Nashových ekvilibrií, $\epsilon$-well-supported Nashovo ekvilibrium.
Hlavní myšlenka je, že $\epsilon$-well-supported Nashovo ekvilibrium je takový smíšený profil, ve kterém každý hráč přiřazuje nenulovou 
pravděpodobnost těm ryzím strategiím, jejichž očekávaný zisk při hraní této ryzí strategie je nejvýše $\epsilon$ od očekávaného zisku při hraní ryzí best response na smíšené strategie ostatních hráčů.

\begin{definition}[\cite{Daskalakis}]
  Mějme $\epsilon > 0$. Smíšený strategický profil $\sigma \in\Delta$ je $\epsilon$-well-supported Nashovo ekvilibrium ve hře $\Gamma$, pokud
  pro všechny $i\in Q$ a $s_i, s_i'\in S_i$ takové, že $\sigma_i(s_i) > 0$, platí $\pi_i(s_i, \sigma_{-i}) \geq \pi_i(s_i', \sigma_{-i}) - \epsilon$.
\end{definition}

\emph{Pozn.  Existuje více formulací definice $\epsilon$-well-supported Nashova ekvilibria (např. jako je uvedeno v~\cite{dp}).}

Jak bylo již zmíněno na začátku kapitoly, výpočet Nashových ekvilibrií je náročný algoritmický problém, proto byla zavedena přibližná 
Nashova ekvilibria. V dalších kapitolách se proto budeme
věnovat algoritmům pro výpočet $\epsilon$-NE a budeme rovněž zkoumat vliv na algoritmickou složitost.

\section{Algoritmus pro výpočet $\epsilon$-Nashova ekvilibria}
V této části se blíže podíváme na algoritmus pro výpočet $\epsilon$-Nashova ekvilibria ve dvouhráčové nekooperativní hře. Tato kapitola je založena na 
článku~\cite{Lipton:2003:PLG:779928.779933} a zaměřuji se zde na hlavní myšlenku (podrobné matematické důkazy jsou k dispozici ve zmíněném článku).

Kromě pojmů popsaných v předchozích kapitolách, je pro další výklad navíc nutné zavést pojem \emph{k-uniformní} smíšená strategie. 
Uvažujme multimnožinu\footnote{může obsahovat několikrát stejnou ryzí strategii} ryzích strategií $S_i'$ hráče $i\in Q$. Smíšená strategie $\sigma_i$ 
hráče $i \in Q$ je potom nazývána \emph{k-uniformní} strategie, pokud $\sigma_i$ je uniformní rozložení pravděpodobnosti na nějaké multimnožině $S_i'$, kde 
$|S_i'| = k$. To znamená, že pokud si pro libovolnou ryzí strategii $s_i \in S_i$ označíme $\alpha_{s_i}$ jako počet výskytů strategie $s_i$ v multimnožině $S_i'$, potom 
$\sigma_i(s_i) = \frac{\alpha_{s_i}}{k}$ (všechny hodnoty vektoru $\sigma_i$ jsou tedy násobky $1/k$).

Předtím, než se budeme věnovat samotnému algoritmu, uvedeme větu, která stojí za celým algoritmem. Uvažujme tedy dále konečnou dvouhráčovou nekooperativní hru v normální formě 
$\Gamma = (Q;S_1,S_2;U_1, U_2)$ a její smíšené rozšíření $\Gamma^m$. Pro jednoduchost zatím uvažujeme, že $|S_1| = |S_2| = n$. Díky strategické ekvivalenci her můžeme 
navíc předpokládat, že hodnoty užitků pro libovolného hráče a libovolnou ryzí strategii, 
jsou v rozmezí 0 -- 1 (dále je taková hra označována jako normalizovaná). Potom platí následující věta.

\begin{theorem}[\cite{Lipton:2003:PLG:779928.779933}]\label{the:eps}
 Pro každé Nashovo ekvilibrium $(\sigma_1^*, \sigma_2^*)$ a pro každé reálné číslo $\epsilon > 0$, existuje pro každé $k \geq \frac{12\ln n}{\epsilon ^2}$
 dvojice $k$-uniformních strategií $\sigma_1'$, $\sigma_2'$ takových, že: 
 \begin{enumerate}
    \item $(\sigma_1', \sigma_2')$ je $\epsilon$-Nashovo ekvilibrium,
    \item $|\pi_1(\sigma_1', \sigma_2') -\pi_1(\sigma_1^*, \sigma_2^*) | < \epsilon$ a $|\pi_2(\sigma_1', \sigma_2') -\pi_2(\sigma_1^*, \sigma_2^*) | < \epsilon$.
 \end{enumerate}
\end{theorem}

Nyní již k samotnému algoritmu, který najde všechna $k$-uniformní $\epsilon$-Nashova ekvilibria ve hře $\Gamma$ a je založen na předchozí větě. Mějme danou hodnotu $\epsilon > 0$. Vstupem algoritmu 
je potom normalizovaná hra $\Gamma$. V prvním kroce nastavíme hodnotu $k = \frac{12\ln n}{\epsilon ^2}$. Dále budeme postupně iterovat přes všechny dvojice $k$-uniformních strategií. Hledání
všech $k$-uniformních strategií hráče $i$ odpovídá vytváření všech možných $k$ prvkových multimnožin z jeho množiny strategií $S_i$. Ke každé nalezené multimnožině přísluší 
$k$-uniformní strategie (viz. definice). Všechny $k$ prvkové multimnožiny nalezneme poměrně jednoduše hrubou silou. Pro každou dvojici 
$k$-uniformních strategií se provede ověření, zda se opravdu jedná o $\epsilon$-Nashovo ekvilibrium. Pro ověření, zda se jedná o $\epsilon$-NE stačí, 
stejně jako v případě smíšeného Nashova ekvilibria, zkoumat jen ty situace, kdy se hráč z profilu \uv{odkloní} ke hře nějaké své ryzí strategie (resp. smíšené strategie, která 
má na pozici $j\in\{1,\dots,n\}$ hodnotu 1). 

Co se týče časové složitosti algoritmu, tak počet $k$ prvkových multimnožin odpovídá počtu kombinací $k$-té třídy z $n$ prvků s opakováním. Pokud si
počet $k$ prvkových multimnožin (resp. počet všech $k$-uniformních strategií) označíme pomocí $P$, tak platí
\begin{equation}
  P = \binom{n+k-1}{k} \sim n^k.
\end{equation}
Vzhledem k tomu, že hledáme dvojice $k$-uniformních strategií, tak máme celkem až $n^{2k}$ možností. Celková asymptotická složitost tedy je 
\begin{equation}
  n^{\mathcal{O}(k)} = n^{\mathcal{O}(\frac{\ln n}{\epsilon^2})}.
\end{equation}
Ověření, zda dvojice $k$-uniformních strategií tvoří $\epsilon$-NE,
se na celkové asymptotické složitosti neprojevý.
Tato složitost se nazývá kvazi-polynomiální.

Věta~\ref{the:eps} nám také říká, že pro každé Nashovo ekvilibrium $\sigma^*$ existuje $k$-uniformní $\epsilon$-NE takové, že očekávaný užitek 
při tomto $\epsilon$-NE pro každého hráče se od očekávaného užitku NE $\sigma^*$ liší nejvýše o $\epsilon$. Výše uvedený algoritmus je 
tedy možné upravit tak, aby pro zadané Nashovo ekvilibrium našel $\epsilon$-NE při kterém se očekávaný užitek obou hráčů liší od oč. 
užitku zadaného NE nejvýše o $\epsilon$ (modifikace by spočívala v přidání omezující podmínky na nalezená $\epsilon$-NE).

Zatím jsme uvažovali, že pro množiny ryzích strategií platí $|S_1| = |S_2| = n$. V případě, že hráči mají různé počty ryzích strategií 
($n_1 = |S_1|$ a $n_2 = |S_2|$), platí výše uvedená věta, jen se změnou podmínky kladené na hodnotu $k$:
\begin{equation}
  k \geq \frac{12\ln \max\{n_1, n_2\}}{\epsilon^2}.
\end{equation}
Algoritmus pro rozdílné počty ryzích strategií zůstává stejný. Asymptotická složitost by zůstala rovněž stejná při uvažování $n = \max\{n_1, n_2\}$.


\section{Hledání přibližných Nashových ekvilibrií pro\\ konstantní $\epsilon$}
V předchozí kapitole jsme se zabývali algoritmem, který pro danou hodnotu $\epsilon$ a nekooperativní normalizovanou dvouhráčovou hru v normální formě $\Gamma$ najde 
$k$-uniformní $\epsilon$-Nashova ekvilibria. Tento algoritmus má kvazi-polynomiální časovou složitost. V této kapitole se
na celou situaci podíváme z trochu jiného pohledu. Budeme se opět zabývat pouze dvouhráčovými nekooperativními normalizovanými hrami v normální formě.
Bude nás ale zajímat existence algoritmů, které v polynomiálním čase naleznou $\epsilon$-Nashovo ekvilibrium pro nějakou pevnou hodnotu $\epsilon$.

\subsection{Výpočet $\frac{1}{2}$-Nashova ekvilibria}
Tato podkapitola je založena na článku~\cite{DASKALAKIS20091581}.
V této podkapitole předpokládáme dvohráčovou nekooperativní normalizovanou hru v normální formě, kde každý hráč má $n$ strategií a zisky jsou zadány ve formě matic 
rozměru $n\times n$ (matice $R$ pro řádkového hráče a $C$ pro sloupcového). Ryzí strategie řádkového hráče odpovídají $n$ řádkům matice
a ryzí strategie sloupcového odpovídají $n$ sloupcům matice. Očekávaný zisk řádkového hráče při hraní smíšeného strategického profilu $(x,y)$
je potom tedy $x^\top Ry$ a podobně očekávaný zisk sloupcového je $x^\top Cy$.

Jak již bylo zmíněno dříve, při ověřování, zda se jedná o Nashovo ekvilibrium (resp. $\epsilon$-NE), stačí kontrolovat zisk jednotlivých hráčů když se 
jednostranně odchýlí k hraní nějaké své ryzí strategie
\footnote{Pro zadanou smíšenou strategii $y^*$ sloupcového hráče je výraz $x(Ry^*)$ v podstatě váženým průměrem hodnot vektoru $Ry^*$ podle smíšené 
strategie $x$. A tedy výraz nabývá maxima pro $x$, které odpovídá nějaké ryzí strategii řádkového hráče (vektor má na určité pozici 1, na ostatních 0). Podobně pro hráče sloupcového.}.
Pojem $\epsilon$-Nashova ekvilibria tedy můžeme ekvivalentně, při použití předchozí notace definovat také následovně. Smíšený profil $(x^*,y^*)$
je $\epsilon$-Nashovo ekvilibrium pro $\epsilon > 0$ ve dvouhráčové nekooperativní hře v normální formě s maticemi zisků $R,C$, pokud platí~\cite{DASKALAKIS20091581}:
\begin{eqnarray}
  \label{eq:eps1}\forall i = 1,\dots,n:\quad e_i^\top Ry^* &\leq& x^{*\top}Ry^* + \epsilon \\
  \label{eq:eps2}\forall i = 1,\dots,n:\quad x^{*\top} Ce_i &\leq& x^{*\top}Cy^* + \epsilon,
\end{eqnarray}
kde $e_i$ ve vektor, jehož $i$-tý prvek je 1 ostatní prvky jsou nastaveny na 0 (odpovídá hraní ryzí strategie $i$).
Algoritmus, který pro dvouhráčovou hru, danou maticemi zisků $R$ a $S$ najde $\frac{1}{2}$-Nashovo ekvilibrium je dán následovně~\cite{DASKALAKIS20091581}.
\begin{enumerate}
 \item Zvol libovolný řádek $i$ pro řádkového hráče (vzhledem k tomu, že ryzí strategie odpovídají řádkům/sloupcům matice, nerozlišuji mezi pojmem strategie a řádek/sloupec).
 \item Nechť $j \in \arg\max_{j'}C_{ij'}$ -- $j$ je tedy strategie sloupcového hráče (sloupec matice), která je best response na strategii (řádek) $i$ řádkového hráče.
 \item Nechť $k \in \arg\max_{k'}R_{k'j}$ -- $k$ je tedy strategie řádkového hráče (řádek matice), která je  je best response na strategii (sloupec) $j$ sloupcového hráče
 \item Ekvilibrium je $(x^*, y^*)$, kde $x^* = \frac{1}{2}e_i + \frac{1}{2}e_k$ a $y^* = e_j$.
\end{enumerate}
Algoritmus pracuje s lineární časovou složitostí (vzhledem k počtu ryzích strategií) -- hledá se jen maximum na řádku a sloupci matice.

Zbývá dokázat, že smíšený profil $(x^*, y^*)$ je opravdu $\frac{1}{2}$-Nashovo ekvilibrium. Očekávaný zisk řádkového hráče 
při hraní $(x^*, y^*)$ je
\begin{eqnarray}
  \pi_1(x^*, y^*) &=& x^{*\top}Ry^* = \left(\frac{1}{2}e_i + \frac{1}{2}e_k\right)^\top Re_j = \left(\frac{1}{2}e_i^\top + \frac{1}{2}e_k^\top\right) Re_j \\ &=& 
  \frac{1}{2}e_i^\top Re_j + \frac{1}{2}e_k^\top Re_j = \frac{1}{2}R_{ij} + \frac{1}{2}R_{kj}.
\end{eqnarray}
Dále z 3. kroku algoritmu víme, že jedno z best response řádkového hráče na strategii $y^*$ je hrát řádek (strategii) $k$. Očekávaný zisk
řádkového hráče při hraní tohoto profilu je $\pi_1(e_k, y^*) = \pi_1(e_k, e_j) = R_{kj}$. Vzhledem k tomu, že $k$ je best response
na $y^*$, řádkový hráč nezíská více při hraní jiné ryzí strategie. Platí tedy
\begin{eqnarray}
  \pi_1(e_k, y^*) - \pi_1(x^*, y^*) = R_{kj} - \left(\frac{1}{2}R_{ij} + \frac{1}{2}R_{kj}\right) = \frac{1}{2}R_{kj} - \frac{1}{2}R_{ij}.
\end{eqnarray}
Vzhledem k tomu, že hra je normalizovaná, předchozí výraz nabývá maxima v případě, kdy $R_{ij} = 0$ a $R_{kj} = 1$. Dále tedy dostáváme
\begin{eqnarray}
  \pi_1(e_k, y^*) - \pi_1(x^*, y^*) \leq \frac{1}{2}R_{kj} \leq \frac{1}{2}.
\end{eqnarray}
První podmínka \eqref{eq:eps1} z definice $\epsilon$-NE je tedy splněna. Nyní budeme dokazovat druhou podmínku.
Zisk sloupcového hráče při hraní $(x^*, y^*)$ je
\begin{equation}
  \pi_2(x^*, y^*) = x^{*\top}Cy^* = \frac{1}{2}C_{ij} + \frac{1}{2}C_{kj}.
\end{equation}
Dále nechť $j'$ je ryzí best response sloupcového hráče na smíšenou strategii $x^*$. Zisk sloupcového hráč při hraní tohoto
smíšeného profilu potom je 
\begin{equation}
  \pi_2(x^*, e_{j'}) = \frac{1}{2}C_{ij'} + \frac{1}{2}C_{kj'}.
\end{equation}
Podobně jako v předchozím případě spočítáme rozdíl očekávaného zisku $\pi_2(x^*, e_{j'})$ od $\pi_2(x^*, y^*)$. Dostáváme tedy
\begin{eqnarray}
  \pi_2(x^*, e_{j'}) - \pi_2(x^*, y^*) &=& \left(\frac{1}{2}C_{ij'} + \frac{1}{2}C_{kj'}\right) - \left(\frac{1}{2}C_{ij} + \frac{1}{2}C_{kj}\right) \\
  &=&\frac{1}{2}\left(C_{ij'} - C_{ij}\right) + \frac{1}{2}\left(C_{kj'}-C_{kj}\right).
\end{eqnarray}
Díky tomu, že $j$ je best response sloupcového hráče na strategii $i$, platí $C_{ij} \geq C_{ij'}$. Opět se snažíme výraz omezit
zhora (což je tedy případ, kdy $C_{ij} = C_{ij'}$) a konečně dostáváme
\begin{equation}
  \pi_2(x^*, e_{j'}) - \pi_2(x^*, y^*) \leq 0 +  \frac{1}{2}(C_{kj'}-C_{kj}) \leq \frac{1}{2}.
\end{equation}
Druhá podmínka \eqref{eq:eps2} z definice $\epsilon$-NE je tedy rovněž splněna. Smíšený profil $(x^*, y^*)$ je tedy $\frac{1}{2}$-NE.


\subsection{Přehled známých výsledků}
V předchozí podkapitole jsme ukázali jednoduchý polynomiální algoritmus pro výpočet $\frac{1}{2}$-NE.
Nicméně existují polynomiální algoritmy, které naleznou přibližná Nashova ekvilibria pro ještě menší hodnoty $\epsilon$. V literatuře se mi podařilo nalézt tyto 
výs\-ledky. V případě $\epsilon$-Nashova ekvilibria existuje pro dvouhráčové nekooperativní normalizované hry
v normální formě polynomiální algoritmus, který nalezne $0.3393$-NE~\cite{Tsaknakis2007}. V případě $\epsilon$-well-supported Nashova ekvilibria existuje pro dvouhráčové nekooperativní normalizované hry
v normální formě polynomiální algoritmus, který nalezne $\frac{2}{3}$-WSNE~\cite{Kontogiannis:2010:WSA:1773434.1773437} (modifikovaná 
verze algoritmu dokonce umožňuje nalézt $(\frac{2}{3} - 0.005913759)$-WSNE~\cite{DBLP:journals/corr/abs-1204-0707}).


\section{Závěr}
V této práci jsem se věnoval problematice Nashova ekvilibria s tolerancí (přibližného Nashova ekvilibria). Zaměřil jsem se hlavně na $\epsilon$-Nashovo ekvilibrium, 
okrajově také na $\epsilon$-well-supported Nashovo ekvilibrium. Tato práce je založena na několika článcích, přičemž jsem se v nich zaměřil převážně na hlavní části
a samotné výsledky. Podrobnější informace včetně důkazů lze najít v originálních článcích. 

První část práce byla věnována základním definicím. Ve druhé části byl popsán algoritmus včetně odvození složitosti algoritmu pro výpočet $\epsilon$-NE v 
normalizovaných dvouhráčových nekooperativních hrách v normální formě. Třetí část byla potom věnována otázce pro jaké hodnoty $\epsilon$ existuje 
polynomiální algoritmus, který nalezne $\epsilon$-NE, resp. $\epsilon$-WSNE.
Na závěr poznamenejme, že kromě zmíněných algoritmů, je možné pro výpočet $\epsilon$-NE použít například i evoluční algoritmy~\cite{Boryczka2013}.

Ačkoliv má použití $\epsilon$-NE i nevýhody, které byly zmíněny v úvodní kapitole, má použití $\epsilon$-NE pozitivní vliv na výpočetní 
složitost, ve srovnání s výpočtem NE. Nicméně i přes všechny dosažené výsledky zůstávají stále některé problémy otevřeny, např. existence
polynomiálního algoritmu pro výpočet $\epsilon$-NE, pro libovolné hodnoty~$\epsilon$~\cite{Nisan:2007:AGT:1296179}.

\bibliographystyle{czplain}
\bibliography{literatura}{}

\end{document}
