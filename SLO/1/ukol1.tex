\documentclass[a4paper,12pt]{article}

\usepackage[utf8]{inputenc}
\usepackage[czech]{babel}
\usepackage[IL2]{fontenc}
\usepackage[left=3cm,text={15cm, 23cm},top=3.5cm]{geometry}

\usepackage{graphicx}
\usepackage{tikz}

\usepackage{amssymb}
\usepackage{amsmath}
\usepackage{amsthm}
\usepackage{forest}
\usepackage[ruled,czech,linesnumbered,noline]{algorithm2e}
\usepackage{array}
\usepackage{multirow}
\usepackage{textcomp}
\usepackage{listings}

\usepackage{xpatch}
\xpretocmd{\algorithm}{\hsize=\linewidth}{}{}
\xpretocmd{\procedure}{\hsize=\linewidth}{}{}

\usepackage{enumitem}
%\usepackage[neverdecrease]{paralist}

\usepackage{pgf}
\usepackage{tikz}
\usetikzlibrary{arrows,automata}

\newcounter{counten}
\setcounter{counten}{1}

\newtheorem{theorem}{Tvrzení}[counten]
\newtheorem{corollary}{Důsledek}[counten]

\lstset{ %
  basicstyle=\footnotesize,
  numbers=left                    % where to put the line-numbers; possible values are (none, left, right)
}


\title{Úkol č. 1 do předmětu Složitost}
\author{Vojtěch Havlena (xhavle03)}
\date{}

\makeatletter
\newcommand{\dotminus}{\mathbin{\text{\@dotminus}}}

\newcommand{\@dotminus}{%
  \ooalign{\hidewidth\raise1ex\hbox{.}\hidewidth\cr$\m@th-$\cr}%
}
\makeatother

\begin{document}

\maketitle

\setlist[enumerate,1]{leftmargin=0.2cm}

\begin{enumerate}[label=\textbf{\arabic*}.]

 \item {\bfseries Příklad}
 Výsledný TS, který rozhoduje jazyk $L = \{a^ib^jc^k | i < j < k\}$ (předpokládám, že $i,j,k \in\mathbb{N}_0$) jsem rozdělil do několika
 částí, které potom mohou být analyzovány samostatně. První částí je dílčí TS $M_1$ (který je zároveň výsledným TS, ale prozatím neuvažujme
 předání řízení stroji $M_2$ úplně na konci). Tento TS ($M_1$) provádí kontrolu, zda řetězec na vstupu je ve tvaru $a^ib^jc^k$, kde 
 $i \in \mathbb{N}_0, j, k \in \mathbb{N}$. Tato kontrola se provádí postupným projitím vstupního řetězce. Je možné, aby řetězec na 
 vstupu neobsahoval prefix tvořený symboly $a$, nicméně vzhledem k tomu,
 jak je zadaný jazyk $L$, v každém případě musí obsahovat sufix $b^jc^k$, kde $j, k > 0$.
 \begin{center}
    \begin{tikzpicture}[->,>=stealth',shorten >=1pt,auto,semithick]
	\tikzstyle{every state}=[text=black,draw=none,inner sep=1pt,minimum size=0pt]

	\node[state,initial,initial text={$M_1:~~~~~~~~~~$}] (A)             {$R$};
	\node[state]         (B) [right of=A,node distance=2cm] {$R$};
	\node[state]         (C) [right of=B,node distance=2cm] {$R$};
	\node[state]         (D) [right of=C,node distance=2cm] {$R$};
	\node[state]         (F) [right of=D,node distance=2cm] {$L_{\Delta}M_2$};
	\node[state]         (E) [right of=B,below of=B,node distance=2cm] {$Error$};

	\path (A) edge  [align=left]            node {$a$} (B)
	      (A) edge  [align=left,bend left=60]            node {$b$} (C)
	      (B) edge  [loop above]            node {$a$} (B)
	      (B) edge  [align=left]            node {$b$} (C)
	      (C) edge  [loop above]            node {$b$} (C)
	      (C) edge  [align=left]            node {$c$} (D)
	      (D) edge  [align=left]            node {$\Delta$} (F)
	      (D) edge  [loop above]            node {$c$} (D)
	      (A) edge  [align=left,below]            node {$c,\Delta$} (E)
	      (B) edge  [align=left,above]            node {$~~c,\Delta$} (E)
	      (C) edge  [align=left,above]            node {$~~a,\Delta$} (E)
	      (D) edge  [align=left,below]            node {$a,b$} (E);
      \end{tikzpicture}
      \end{center}
   Druhou částí je dílčí TS $M_2$ (opět neuvažujme předání řízení stroji $M_3$). Tento stroj postupně po jednom odstraňuje (pomocí posuvu vlevo, $S_L$) symboly $a$, $b$ a $c$, 
   dokud zbývá nějaký symbol $a$. V případě, že úplně na začátku první symbol je $b$, předáme řízení stroji $M_3$ (řetězec neobsahuje žádný symbol $a$). Pokud odstraníme symbol $a$ a neexistuje další symbol $b$ nebo $c$, který bychom mohli odstranit, řetězec odmítneme.
   Stejně tak odmítneme, pokud s odstraněním posledního symbolu $a$, zbydou na pásce pouze symboly $c$ nebo pouze $\Delta$ (počet některých symbolů je stejný). 
   
   Více detailně: TS $M_2$, pokud je první vstupní symbol $a$,
   tento symbol odstraní posuvem užitečného obsahu pásky doleva. Dále se posunuje doprava, dokud nenarazí na jiný znak než $a$. 
   V případě, že další znak je $c$ nebo $\Delta$, je řetězec odmítnut. Když narazíme na symbol $b$, opět pomocí posuvu už. obsahu pásky vlevo
   smažeme symbol $b$ a opět přejdeme zbylé symboly $b$ (pokud dalším symbolem je $\Delta$ a ne $c$, odmítneme). V případě, že narazíme na symbol $c$,
   odstraníme i tento symbo posuvem obsahu vlevo a posuneme se vpravo na první pozici. Pokud je první symbol modifikovaného řetězce $a$, celý tento 
   postup opakujeme. Pokud prvním symbolem je $c, \Delta$, počet nekterých symbolů je stejný, proto odmítneme. Jinak jsme odstranili všechny symboly
   $a$ a předáme řízení stroji $M_3$.
   \begin{center}
    \begin{tikzpicture}[->,>=stealth',shorten >=1pt,auto,semithick]
	\tikzstyle{every state}=[text=black,draw=none,inner sep=1pt,minimum size=0pt]

	\node[state,initial,initial text={$M_2:~~~~~~~~~~$}] (A)             {$R$};
	\node[state]         (B) [right of=A,node distance=2cm] {$S_L$};
	\node[state]         (M) [below of=A,node distance=2cm] {$M_3$};
	\node[state]         (C) [right of=B,node distance=2cm] {$S_L$};
	\node[state]         (D) [right of=C,node distance=2cm] {$S_L$};
	\node[state]         (DD) [right of=D,node distance=0.5cm] {$L_\Delta$};
	\node[state]         (DDD) [right of=DD,node distance=0.4cm] {$R$};
	\node[state]         (F) [right of=DDD,node distance=2cm] {$M_3$};
	\node[state]         (G) [right of=B,below of=B,node distance=1cm] {$R_{\neg a}$};
	\node[state]         (H) [right of=C,below of=C,node distance=1cm] {$R_{\neg b}$};
	\node[state]         (E) [below of=C,node distance=2.5cm] {$Error$};

	\path (A) edge  [align=left]            node {$a$} (B)
	      (A) edge  [align=left]            node {$b$} (M)
	      (B) edge  [align=left]            node {$b$} (C)
	      (C) edge  [align=left]            node {$c$} (D)
	      (B) edge  [below]            node {$a$} (G)
	      (G) edge  [below]            node {$b$} (C)
	      (C) edge  [below]            node {$b$} (H)
	      (H) edge  [below]            node {$c$} (D)
	      (B) edge  [below,bend right=50]            node {$c,\Delta~~$} (E)
	      (G) edge  [below]            node {$c,\Delta~~$} (E)
	      (C) edge              node {$\Delta$} (E)
	      (H) edge  [below]            node {$\Delta$} (E)
	      (DDD) edge  [above,bend left=50]            node {$c,\Delta~~$} (E)
	      (DDD) edge  [align=left,bend right=40,above]            node {$a$} (B)
	      (DDD) edge  [align=left]            node {$b$} (F);
      \end{tikzpicture}
      \end{center}
    V případě, když už modifikovaný řetězec neobsahuje žádné symboly $a$, řízení se předá stroji $M_3$. Tento stroj
    postupně odstraňuje symboly $b$ a $c$ posuvem obsahu pásky vlevo. Opět se kontroluje, zda je počet symbolů $b$ větší než počet symbolů $c$. 
    V případě, kdy se odstraní všechny symboly $b$
    a zbyly nějaké symboly $c$, tyto symboly se odstraní a řetězec je přijat.
      \begin{center}
    \begin{tikzpicture}[->,>=stealth',shorten >=1pt,auto,semithick]
	\tikzstyle{every state}=[text=black,draw=none,inner sep=1pt,minimum size=0pt]

	\node[state,initial,initial text={$M_3:~~~~~~~~~~$}] (A)             {$S_L$};
	\node[state]         (D) [right of=A,node distance=2cm] {$S_L$};
	\node[state]         (DD) [right of=D,node distance=0.5cm] {$L_\Delta$};
	\node[state]         (DDD) [right of=DD,node distance=0.4cm] {$R$};
	\node[state]         (H) [right of=A,below of=A,node distance=1cm] {$R_{\neg b}$};
	\node[state]         (G) [right of=DDD,node distance=2cm] {$S_L$};
	\node[state]         (E) [below of=H,node distance=1.5cm] {$Error$};
	\node[state]         (S) [right of=G,node distance=2cm] {$Succ$};

	\path 
	      (H) edge  [align=left,below]            node {$c$} (D)
	      (A) edge  [align=left]            node {$b$} (H)
	      (A) edge  [align=left]            node {$c$} (D)
	      (A) edge  [below,bend right]            node {$\Delta$~~~} (E)
	      (H) edge  [below]            node {$\Delta$} (E)
	      (DDD) edge  [below,bend left]            node {$~~\Delta$} (E)
	      (DDD) edge  [align=left,bend right=40,above]            node {$b$} (A)
	      (DDD) edge  [align=left]            node {$c$} (G)
	      (G) edge  [loop above]            node {$c$} (G)
	      (G) edge              node {$\Delta$} (S);
      \end{tikzpicture}
      \end{center}
     \emph{Poznámky:} Nepřijetí řetězce je v diagramu pro jednoduchost označeno jako $Error$, což může označovat posouvání
     hlavy doleva až do abnormálního zastavení. Dále v diagramu se označuje $S_L$ jako posuv užitečného obsahu pásky, který se 
     vyskytuje napravo od aktuální pozice hlavy, o jednu pozici vlevo. V případě, že se vpravo od aktuálního symbolu vyskytuje
     symbol $\Delta$, je aktuální symbol přepsán na $\Delta$.
     
     {\bfseries Horní odhad časové složitosti}
     Vzhledem k tomu, jak je výsledný TS rozdělen na dílčí části, časová složitost vzhledem ke vstupu $w$ je dána jako
     $$t_M(w) = t_{M_1}(w) + t_{M_2}(w) + t_{M_3}(w).$$
     Opět uvažujme že jednotlivé dílčí stroje neobsahují předání řízení dalšímu stroji $M_i$ a výsledný TS $M$ je dán
     kompozicí jednotlivých strojů: $M: \rightarrow M_1M_2M_3$. Tedy 
     $$
      T_M(n) = \max\{t_{M_1}(w) + t_{M_2}(w) + t_{M_3}(w) | w\in\Sigma^n\}.
     $$
     Uvažujme libovolný řetězec $w$ délky nejvýše $n$. Potom $t_{M_1}(w) \leq 2n + 2$ -- vstupní řetězec se celý projde
     a pak se provede návrat na 1. pozici pásky pomocí $L_\Delta$. Tedy $$t_{M_1}(w) \in O(n)$$
     
     Složitost posuvu obsahu pásky vlevo $S_L$ je $O(n)$ -- postupně
     čteme symboly vlevo a zapisujeme je na pozici napravo, tedy v nejhorším případě projdeme celý řetězec $(n)$ krát nějaká kladná
     konstanta kvůli posunování se na předchozí pozici a zapisování posouvaného symbolu. Složitost jednoho průchodu stroje $M_2$ 
     (bez zpětné hrany $a$) je tedy $3t_{S_L} + t_{R_{\neg a}} + t_{R_{\neg b}} + t_{L_\Delta} + c$, kde $c\in\mathbb{N}$. Do konstanty
     $c$ jsou schovány jednotlivé posuvy, např. $R,Error$. A tento jeden průchod se provede maximálné $i$-krát (počet symbolů $a$). Vzhledem
     k tomu, že stroj $M_1$ provede kontrolu tvaru vstupního řetězce, můžeme dále uvažovat vstupní řetězec ve tvaru $a^ib^jc^k$. Tedy
     v nejhorším případě máme 
     $$
      t_{M_2}(w) \leq i(3t_{S_L} + t_{R_{\neg a}} + t_{R_{\neg b}} + t_{L_\Delta} + c) \leq n(3t_{S_L} + 2n + c),
     $$
     což spolu s horním odhadem $S_L$ a tím, že $t_{R_{\neg a}} + t_{R_{\neg b}} \leq n$ a $t_{L_\Delta} \leq n + c$, 
     dává $t_{M_2}(w) \in O(n^2)$.
     
     Podobně složitost v nejhorším případě jednoho průchodu první částí stroje $M_3$ (bez zpětné hrany $b$) je $2t_{S_L} + t_{R_{\neg b}} + t_{L_\Delta} + c$ a
     tento průchod se provede celkem $(j - i)$-krát (v případě vykonávání stroje $M_3$ už víme, že $j > i$ a v případě vykonávání poslední 
     části stroje $M_2$ víme také, že $k > j$). Tedy společně s druhou částí stroje $M_3$ dostáváme
     $$
      t_{M_3}(w) \leq (j - i)(2t_{S_L} + t_{R_{\neg b}} + t_{L_\Delta} + c) + (k - j)t_{S_L} \leq n(3t_{S_L} + 2n + c).
     $$
     Tím pádem opět se složitostí $S_L$ dostáváme $t_{M_3}(w) \in O(n^2)$. Celková časová složitost v 
     nejhorším případě je tedy $T_M(n) \in O(n^2)$.
     
     {\bfseries Horní odhad prostorové složitosti}
     Vzhledem k tomu, jak je Turingův stroj navržen se využívá pouze ta část pásky, kde je zapsán vstup. Symboly se pouze odstraňují posunem obsahu pásky
     vlevo. Hlava se dostane nejdál na pozici 1 za vstupním řetězcem $w$, tj. $n+1$, kde $n = |w|$. Tedy celkový horní 
     odhad prostorové složitosti je $S_M(n) \in O(n)$.

   \item {\bfseries Příklad}
   Výsledný RAM program je možné rozdělit do dvou částí. První část programu, $sqr(n)$, počítá druhou mocninu čísla $n$. Druhá část s využitím 
   $sqr$ počítá druhou odmocninu. První verze programu zmenšuje postupně číslo $n$ a hledá největší $i$ takové, že $i^2 \leq n$. Tedy algoritmus v 
   pseudokódu vypadá následovně.
   
    \begin{algorithm}[H]
 \SetKwInput{Input}{Vstup}\SetKwInOut{Output}{Výstup}
 \SetNlSty{}{}{:}
 \SetNlSkip{-1.0em}
 \SetInd{0.5em}{0.5em}
 \Input{Číslo $n$ takové, že $n \geq 0$}
 \Output{$\lfloor\sqrt{n}\rfloor$}
 \BlankLine
 \Indentp{1.4em}
    $i \gets n$ \\
    $j \gets sqr(i)$ \\
    \While{$j > n$} {
      $i \gets i - 1$ \\
      $j \gets sqr(i)$ \\
    }
    \Return $i$
 \caption{\textsc{Základní schéma algoritmu}}
 \end{algorithm}
  
  Nevýhoda tohoto algoritmu je, že se v cyklu opakovaně počítá druhá mocnina čísla $i$. Proto je možné využít toho, že
  $(i-1)^2 = i^2 - 2i + 1$. Novou hodnotu $j$ je tedy možné spočítat z předchozí hodnoty. Druhá mocnina se tedy spočítá
  pouze jednou na začátku programu. Navíc násobení $2i$ lze nahradit za sčítání $i + i$. V cyklu se tedy používají pouze
  operace $+,-$. Zápis výsledného algoritmu v pseudokódu je následující. Na řádcích 5 -- 8 se postupným přičítáním čísla $i$ 
  počítá druhá mocnina čísla $i$.
  Pro ušetření počtu registrů se při výpočtu druhé mocniny využívají proměnné (registry), které se potom dále používají pro 
  výpočet druhé odmocniny.
  
  \begin{algorithm}[H]
 \SetKwInput{Input}{Vstup}\SetKwInOut{Output}{Výstup}
 \SetNlSty{}{}{:}
 \SetNlSkip{-1.0em}
 \SetInd{0.5em}{0.5em}
 \Input{Číslo $n$ takové, že $n \geq 0$}
 \Output{$\lfloor\sqrt{n}\rfloor$}
 \BlankLine
 \Indentp{1.4em}
    $i \gets n$ \\
    $k \gets 0$ \\
    $l \gets 0$ \\
    $j \gets 0$ \\
    \While{$k < i$} {
      $j \gets j + i$ \\
      $k \gets k + 1$
    }
    \While{$j > n$} {
      $k \gets i$ \\
      $i \gets i - 1$ \\
      $l \gets k + k$ \\
      $j \gets j - l$  \\
      $j \gets j + 1$ \dots ($j \gets j - 2k + 1$)
    }
    \Return $i$
 \caption{\textsc{Výsledný algoritmus}}
 \end{algorithm}
 
 Pro výpočet jsou tedy potřeba registry: $i_1$ (vstup), $r_0$ (akumulátor), $r_1$ (proměnná $i$), 
 $r_2$ (proměnná $k$), $r_3$ (proměnná $j$), $r_4$ (proměnná $l$).

 \begin{center}
 \begin{minipage}[t]{.3\textwidth}
\begin{lstlisting}
READ 1

.sqrstart
LOAD 1
SUB 2
JZERO .cstart
JNEG .cstart
LOAD 3
ADD 1
STORE 3
LOAD 2
ADD =1
\end{lstlisting}%
\end{minipage}%
\hfill
\begin{minipage}[t]{.3\textwidth}
\begin{lstlisting}[firstnumber=13]
STORE 2
JUMP .sqrstart

.cstart
READ 1
SUB 3
JZERO .halt
JNEG .halt
LOAD 1
STORE 2
SUB =1
STORE 1
LOAD 2
\end{lstlisting}%
\end{minipage}%
\hfill
\begin{minipage}[t]{.3\textwidth}
\begin{lstlisting}[firstnumber=26]
ADD 0
STORE 4
LOAD 3
SUB 4
ADD =1
STORE 3
JUMP .cstart
.halt
LOAD 1
HALT
\end{lstlisting}%
\end{minipage}%
\end{center}

  {\bfseries Uniformní časová a prostorová složitost}
  Předpokládejme vstupní vektor $I = (n)$. Kód v pseudokódu na řádcích 5 -- 8 (v RAM programu to odpovídá řádkům 4 -- 14) se provede 
  celkem $n$-krát ($n$-krát se přičte číslo $n$). Tedy složitost tohoto úseku je celkem $11n \in O(n)$. Kód v pseudokódu na
  řádcích 9 -- 15 (v RAM programu 17 -- 32) se provádí jednou pro každé $i \in \{n, \dots \lfloor\sqrt{n}\rfloor\}$. Cyklus
  se tedy bude provádět celkem $(n - \sqrt{n} + 1)$-krát. Celková složitost tohoto cyklu je tedy $16(n - \sqrt{n} + 1) \leq 16n\in O(n)$.
  Celkově tedy pro vstup $I = (n)$ máme 
  $$
   t^{uni}_{\Pi}(I) \leq 27n + c \in O(n),
  $$
  kde $c\in \mathbb{N}$ a tato konstanta schovává jednotlivé kroky (které nejsou v cyklu) jako počáteční čtení vstupu, zastavení, \dots.
  Délka vstupu $I = (n)$ je $len(I) = \lceil\log_2(n)\rceil$. Horní odhad uniformní celkové časové složitosti $T_\Pi$ tedy bude:
  $$
    T_\Pi (k) \leq \max\left\{t_\Pi^{uni}(I) | len(I) = k\right\} = \max\left\{t_\Pi^{uni}(I) | \lceil\log_2(n)\rceil = k\right\} \in O(2^k).
  $$
  (délka vstupu je $n \leq 2^k$, složitost $t^{uni}_{\Pi}(I)$ je $O(n)$, tudíž horní odhad $T_\Pi (k)$ je $O(2^k)$).
  Co se týče uniformní prostorové složitosti pro vstup $I = (n)$, $s^{uni}_{\Pi}(I) = 6 \in O(1)$ (během výpočtu se použije max. 6 registrů).
  Horní odhad uniformní celkové prostorové složitosti $S_\Pi$ tedy bude
  $$
    S_\Pi (k) = 6 \in O(1).
  $$
   
  {\bfseries Logaritmická časová a prostorová složitost}
  V případě logaritmické časové složitosti nejprve spočítáme horní odhad cenové funkce pro vstup $I = (n)$. Největší
  číslo se kterým se v programu pracuje je $n^2$ (předpokládejme, že $n^2 > c_0$, kde $c_0$ je maximální konstanta použitá v programu. Pokud 
  by platilo $n^2 < c_0$, tak horní odhad cenové funkce bude právě tato konstanta, nicméně na asymptotické složitosti to nic nemění). Tedy 
  pro každý krok programu $k$:
  $$
    c^{log}_{(\Pi,I)}(k) \leq len(n^2) = \lceil\log_2(n^2)\rceil \leq 2\lceil\log_2(n)\rceil
  $$
  Potom tedy horní odhad log. časové složitosti pro vstup $I = (n)$ je 
  $$
    t^{log}_{\Pi}(I) \leq t^{uni}_\Pi (I)c^{log}_{(\Pi,I)} \in O(n\log_2(n))
  $$
  a horní odhad logaritmické celkové časové složitosti $T_\Pi$ tedy bude
  $$
    T_\Pi (k) \leq \max\left\{t_\Pi^{log}(I) | \lceil\log_2(n)\rceil = k\right\} \in O(2^kk).
  $$
  Co se týče logaritmické prostorové složitosti, největší uložená hodnota pro vstup $I = (n)$ je $n^2$, tedy horní odhad
  logaritmické prostorové složitosti je (celkem je 6 registrů)
  $$
    s^{log}_{\Pi}(I) \leq 6\cdot2\lceil\log_2(n)\rceil \in O(\log_2(n)).
  $$
  Horní odhad logaritmické prost. složitosti tedy bude
  $$
    S_\Pi (k) \leq \max\left\{s_\Pi^{log}(I) | \lceil\log_2(n)\rceil = k\right\} \in O(k).
  $$

 \enlargethispage{5em}
  \item {\bfseries Příklad}
  
   Nejprve důkaz toho, že $L \in NTIME(n)$. V příkladu uvažuji následující definici $NTIME$ (viz. TIN):
   $$
    NTIME(f(n)) = \{L | \mbox{existuje k-páskový NTS } M: L = L(M) \mbox{ a }T_M \in O(f(n)) \}
   $$
   Nechť $L$ je bezkontextový jazyk. Potom k němu existuje gramatika $G = (N, \Sigma, P, S)$ v Chomského normální formě taková, že 
 $L(G) = L$. Sestavíme 3-páskový NTS $M$, který bude simulovat provádění pravidel v CNF z množiny $P$.
 \begin{itemize}
  \item[--] První páska obsahuje vstupní řetězec $w$.
  \item[--] Druhá páska slouží pro simulaci nejlevější derivace. V průběhu výpočtu obsahuje řetězec, který je celý 
  tvořen symboly z $\Sigma$ a poslední symbol je z $N$.
  \item[--] Třetí páska slouží jako zásobník pro neterminály, které jsou součástí větné formy, ale nejsou nejlevější. Kompletní větnou formu 
  lze sestavit konkatenací užitečného obsahu 2. pásky s užitečným obsahem 3. pásky v reverzním pořadí.
\end{itemize}
NTS $M$ potom pracuje tak, že se nejprve na 2. pásku vloží symbol $S$ a na pozici $n + 1$ na 2. a 3. pásce se vloží symbol, který
slouží jako okraj (nikdy nemůžeme mít delší větnou formu než $n$). Tento okrajový symbol nemůže být přepsán (pokud by měl být přepsán, NTS $M$
abnormálně zastaví).
\begin{itemize}
  \item[--] Nechť $A$ je neterminál na 2. pásce. Potom se opakovaně nedeterministicky zvolí pravidlo $A\rightarrow \alpha \in P$.
  Pokud $\alpha = BC$, $B, C \in N$, tak se aktuální neterminál na 2. pásce přepíše na $B$ a do zásobníku na 3. pásce se vloží neterminál $C$.
  Pokud $\alpha = b$, $b\in\Sigma$, tak se přepíše symbol neterminálu na 2. pásce za $b$, hlava se posune o pozici doprava a zapíše se neterminál, 
  který je na vrcholu zásobníku na 3. pásce (symbol, který je na pozici hlavy na 3. pásce). Na 3. pásce je potom tento symbol vymazán.
  \item[--] NTS $M$ nakonec zkontroluje, zda obsah 2. pásky je shodný s obsahem 1. pásky (tedy jestli byl vygenerován řetězec $w$). Pokud ano, zastaví
  přechodem do $q_F$. V opačném případě abnormálně zastaví.
 \end{itemize}

 Co se týče časové složitosti (počtu kroků NTS $M$), tak přepis symbolů, vkládání a 
 výběr ze zásobníku lze provést v konstantním počtu kroků (omezeno nějakou kladnou konstantou $k$, 
 nezávislé na velikosti vstupního řetězce). Pro kontrolu, zda je obsah 2. pásky
 shodný s obsahem 1. pásky je potřeba $O(n)$ kroků (toto se ale provede pouze jednou během výpočtu). Vzhledem k 
 tomu, že gramatika je CNF, tak počet použitých pravidel $p$ pro vygenerování věty o délky $n$ je $p = 2n - 1 \in O(n)$.
 Navíc kvůli omezení na užitečnou délku 2. a 3. pásky NTS skončí i v případě odmítnutí s počtem kroků $O(n)$.
 Celkový počet kroků je tedy vzhledem k výše uvedenému $O(n)$ a tedy $L\in NTIME(n)$. 
 
 \enlargethispage*{5em}
 Vzhledem k tomu, že funkce $f(n) = n$ je časově i prostorově zkonstruovatelná, přičemž $f(n) \geq \log_2(n)$, platí následující vztah
 $$
  NTIME(f(n)) \subseteq DSPACE(f(n)).
 $$
 Tím pádem platí $L \in DSPACE(n)$.
 
  
  \end{enumerate}



\end{document}