\documentclass[a4paper,12pt]{article}

\usepackage[utf8]{inputenc}
\usepackage[czech]{babel}
\usepackage[IL2]{fontenc}
\usepackage[left=3cm,text={15cm, 23cm},top=3.5cm]{geometry}

\usepackage{graphicx}
\usepackage{tikz}

\usepackage{amssymb}
\usepackage{amsmath}
\usepackage{amsthm}
\usepackage{forest}
\usepackage[ruled,czech,linesnumbered,noline]{algorithm2e}
\usepackage{array}
\usepackage{multirow}
\usepackage{textcomp}
\usepackage{listings}

\usepackage{xpatch}
\xpretocmd{\algorithm}{\hsize=\linewidth}{}{}
\xpretocmd{\procedure}{\hsize=\linewidth}{}{}

\usepackage{enumitem}
%\usepackage[neverdecrease]{paralist}

\usepackage{pgf}
\usepackage{tikz}
\usetikzlibrary{arrows,automata}

\newcounter{counten}
\setcounter{counten}{1}

\newtheorem{theorem}{Tvrzení}[counten]
\newtheorem{corollary}{Důsledek}[counten]

\lstset{ %
  basicstyle=\footnotesize,
  numbers=left                    % where to put the line-numbers; possible values are (none, left, right)
}


\title{Úkol č. 2 do předmětu Složitost}
\author{Vojtěch Havlena (xhavle03)}
\date{}

\makeatletter
\newcommand{\dotminus}{\mathbin{\text{\@dotminus}}}

\newcommand{\@dotminus}{%
  \ooalign{\hidewidth\raise1ex\hbox{.}\hidewidth\cr$\m@th-$\cr}%
}
\makeatother

\begin{document}

\maketitle

\setlist[enumerate,1]{leftmargin=0.2cm}

\begin{enumerate}[label=\textbf{\arabic*}.]

 \item {\bfseries Příklad}\\
  \textbf{Časová složitost} Pro jednoduchost uvažujme, že $n = |V|$. Cyklus na řádcích 2. -- 3. se provede celkem $n$-krát. 
  Zápis položky do pole se provede s konstantní složitostí. Pro kód na řádcích 2. -- 3. máme tedy celkem časovou složitost $O(n)$.
  Kód na řádku 4. (zápis do pole) se provede s konstantní složitostí.
  
  Co se týče složistí práce se zásobníkem, tak uvažuji implementaci zásobníku v podobě jednosměrného vázaného seznamu s ukazatelem
  na vrchol zásobníku. Vytvoření je tedy možné v konst. čase (přiřazení ukazateli na vrchol zás. hodnotu \texttt{NULL}). Přidání prvku
  je možné také v kons. čase (vytvoření nového prvku, nastavení ukazatele na vrchol, nastavení uk. vrcholu zásobníku na nový prvek).
  Podobně odebrání prvku je možné také v konst. čase (vrchol nastavím na další prvek, smažu prvek na který ukazoval vrchol před tím).
  Zjištění, zda je zás. neprázdný lze opět provést v konst. čase (porovnání jestli vrchol zás = \texttt{NULL}).
  
  Kód na 5. řádku se tedy provede s časovou složitostí $O(1)$. Vzhledem k tomu, že do zásobníku jsou vkládány vrcholy, které mají
  stav nastavený na \texttt{FRESH} a před samotným vložením do zásobníku jsou nastaveny na \texttt{OPEN} a už nikdy nejsou nastaveny zpět na \texttt{FRESH}, každý
  vrchol je vložen do zásobníku nejvýše jednou (v případě nesouvislého grafu tam nemusí být některé vrcholy vloženy vůbec). 
  Dále uvažuji, že vstupní 
  graf je zadán seznamem sousedů a pokud si označím $Adj[v]$ jako seznam vrcholů, které jsou s vrcholem $v$ spojeny hranou, platí: 
  $$
    \sum_{v\in V} |Adj[v]| \leq 2|E|
  $$
  Cyklus na
  řádcích 8. -- 11. se tedy provede celkem až $2|E|$-krát (v případě orientovaných grafů až $|E|$-krát) -- 
  Kód na řádcích 9. -- 11. se provede s konst. složitostí. Nakonec výběr
  ze zásobníku na ř. 7. a nastavení stavu na ř. 12. se tedy celkem provede až $n$-krát. Díky tomu, že $|E| \leq n^2$ (rovnost nastává pro 
  orientované úplné grafy), je celková složitost vzhledem k počtu uzlů $n$: $O(n + n + |E|) = O(n +n + n^2) = O(n^2)$.
  
  \textbf{Prostorová složitost} 
  V algoritmu jsou využity 3 pole (\texttt{state}, \texttt{d}, \texttt{p}) o velikosti počtu vrcholů grafu $n$. Jak bylo uvedeno výše, do zásobníku je vložen
  každý vrchol nejvýše jedenkrát, velikost zásobníku je tedy v nejhorším případě $n$. Celková prost. složitost zásobníku bude
  $a(n + 1)$, kde $1$ je tam pro vrchol zásobníku a $a$ v sobě zahrnuje prost. složitost ukazatele na další prvek a místo pro 
  uložení hodnoty. Celková prostorová složitost tedy bude $O(n)$.
  
 \item {\bfseries Příklad}
  Nejprve důkaz toho, že $EXACT\_LENGTH\in NP$. Pro zadanou instanci $(G, X)$ bude NTS $M$ nedeterministicky
  tvořit cestu v $G$ (stačí pouze posloupnost navzájem různých vrcholů mezi kterými existuje hrana, nejedná se o multigraf). Těchto vrcholů
  na dané cestě může být nejvýše $|V| = n$. 
  Generování cesty podrobněji probíhá takto: Nejprve se nedeterministicky zvolí počáteční uzel. Poté se ned. zvolí jestli se bude 
  pokračovat v generování. Pokud ano, ned. se zvolí uzel, do kterého vede z předchozího uzlu hrana (procházení množiny hran, složitost $O(|E|) = O(n^2)$. 
  Tento postup se opakuje maximálně do vygenerování posloupnosti $n$ vrcholů a $n-1$ hran. Dále se zkontroluje, zda se vygenerované uzly neopakují (složitost $O(n^2)$).
  Pokud se opakují, $M$ zamítne.
  NTS $M$ nakonec jen sečte váhy hran a porovná s hodnotou $X$. Pokud
  se shoduje, přijme, jinak zamítne. Uvedený NTS $M$ tedy pracuje v polynomiálním čase.
  
  
  Redukci v tomto případě provedu z problému $SUBSET\_SUM$. Předpokládejme, že $(S, w, W)$, kde $S$ je konečná
  množina prvků, $w$ je váhová funkce $w: S\rightarrow \mathbb{Z}$ a $W$ je požadovaná váha. Převod na problém
  $EXACT\_LENGTH$ provedeme následovně (redukce $R$). Každá hrana v grafu bude mít váhu jako některý prvek z $S$. Vzhledem k tomu, 
  že se v problému $EXACT\_LENGTH$ hledá orientovaná cesta (nemůžou se tedy opakovat vrcholy),
  budeme potřebovat alespoň $|S| = n$ hran (tedy $n+1$ vrcholů), abychom byli schopni popsat váhu celé množiny $S$.
  Budeme také uvažovat uspořádání na množině vrcholů. Pomocí tohoto uspořádání budeme mj. definovat množinu hran. 
  Každou podmnožinu $S$ budeme potom uvažovat v seřazené formě a bude pro ni existovat odpovídající cesta v grafu.
  
  Tedy $V = S\cup \{p\}$, kde $p\notin S$. Dále uvažujme libovolné lineární uspořádání $\preceq$ na $V$, takové
  že $\forall x\in V: p\preceq x$ ($p$ je nejmenší prvek, vzhledem k tomuto uspořádání, $p = \min_\preceq V$).
  Množinu hran potom sestavíme následovně:
  $$
    E = \{ (u,v)\ |\ u, v \in S, u\preceq v, u\neq v \} \cup \{ (p, v') \},
  $$
  kde $v' = \min_\preceq S$. Pro vrchol $p$ existuje pouze orientovanou hranu do nejmenšího prvku množiny 
  $S$ (vzhledem k $\preceq$). Co se týče hodnotové funkce, tak ta bude vypadat:
  $$
    d: E\rightarrow \mathbb{Z},\quad (u,v) \mapsto w(v).
  $$
  Tato funkce je dobře definovaná, protože pro všechny vrcholy, mimo $p$, je definována funkce $w$, ale
  do vrcholu $p$ nevede žádná hrana. Nakonec $X = W$.
  
  Velikost výsledného grafu bude $|V| = n + 1$ a 
  $$
    |E| = 1 + n + (n-1) + \cdots + 1 = 1+ \frac{n(n-1)}{2} \in O(n^2).
  $$
  Redukci $R$ je tedy možné implementovat deterministickým TS v polynomiálním čase (Za uspořádání $\preceq$ lze
  například uvažovat pořadí, v jakém byly prvky z $S$ zapsány na pásku). Zbývá dokázat, že 
  $w\in SUBSET\_SUM \Leftrightarrow R(w) \in EXACT\_L\-ENGTH$.
  \begin{itemize}
   \item[$(\Rightarrow)$] Pokud $(S, w, W)\in SUBSET\_SUM$, potom existuje $S'\subseteq S$ takové, že $w(S') = W$.
   Nechť $\preceq$ je zvolené lin. uspořádání a $q = \min_\preceq S'$. Dále nechť $q'\in V$ je bezprostřední předchůdce
   $q$ (tedy $q'\preceq q $ a $\forall q''\in V: q' \preceq q'' \Rightarrow q \preceq q''$) (takový určitě existuje 
   vzhledem k přidanému vrcholu $p$, který není v $S$). Dále si prvky z $S'$ seřadíme do rostoucí posloupnosti 
   $(s_1, s_2, \dots, s_m)$ vzhledem k $\preceq$. Potom ale v grafu $G$ existuje cesta $\pi = q'e_1s_1e_2\dots e_ms_m$ a
   $$
    len(\pi) = \sum_{i = 1}^md(e_i) = \sum_{i = 1}^m w(s_i) = W = X.
   $$
   
   \item [$(\Leftarrow)$] Nechť $(G, X)\in EXACT\_LENGTH$. Potom existuje cesta $\pi = v_1e_1\dots e_{m-1}v_m$ taková,
   že $len(\pi) = X$. Vzhledem k tomu, jak je graf $G$ definován, platí, že $S' = \{ v_2, \dots v_m\} \subseteq S$ a
   $$
    \sum_{i = 2}^m w(v_i) = \sum_{i = 1}^md(e_i) = len(\pi) = X = W.
   $$
  \end{itemize}
  \emph{Pozn.} Implicitně předpokládám, že v případě nesprávně zformovaného vstupu redukce $R$ zamítne.
  
 \item {\bfseries Příklad}
  Nejprve důkaz, že $L_t \in NP$. Existuje DTS, který přijímá $L_t$ (zkontroluje, zda je na vstupu 0, pokud ano 
  přijme, jinak odmítne) -- to je možné provést v konstantním čase. Vzhledem k tomu, že DTS je speciální případ NTS,
  $L_t \in NP$.
  
  Dále ukážeme, že pro každý jazyk $L' \in NP$ platí $L' \leq L_t$. Nechť $L'$ je libovolný jazyk z $NP$. Vzhledem
  k tomu, že $P = NP$ (předpoklad), existuje DTS $M$ pracující s polynomiální časovou složitostí $p(n)$ takový, že $L(M) = L'$.
  Zkonstruujeme redukci $R$, která pracuje následovně: $R$ na svém vstupu simuluje DTS $M$. Pokud $M$ přijme, potom
  $R$ smaže obsah výstupní pásky a zapíše symbol 0. Pokud $M$ odmítne, $R$ na pásku zapíše symbol 1.
  
  Vzhledem k tomu, že $M$ je DTS, je možné i redukci $R$ implementovat pomocí DTS $M_R$ (simulaci lze provádět
  deterministicky). Co se týče časové složitosti $M_R$, tak pro simulaci $M$ je potřeba $O(p(n))$ kroků, kde $p(n)$ je polynom. 
  Smazání pásky zabere v nejhorším případě $O(p(n))$ kroků a zápis 0 nebo 1 se provede v konstantním čase.
  Celková složitost stroje $M_R$ je tedy $O(p(n))$, to znamená polynomiální.
  
  Navíc dostáváme: $w\in L' \Rightarrow R(w) = 0 $ a tedy $ R(w) \in L_t$. Také
  $w\notin L' \Rightarrow R(w) = 1 $ a tedy $ R(w) \notin L_t$. A tím pádem $w\in L' \Leftrightarrow R(w) \in L_t$.
  Jazyk $L_t$ je tedy $NP$-úplný.
  
 \item {\bfseries Příklad} 
  Předpokládejme, že $P = NP$. Dále pro každý jazyk $L\in NP$, který není prázdný nebo univerzální, existují řetězce $w_1$ a $w_2$
  takové, že $w_1 \in L$ a $w_2\notin L$. Potom je možné použít pro důkaz, že $L$ je $NP$-úplný, důkaz předchozího tvrzení s tím, že 
  řetězec 0 nahradíme za $w_1$ a řetězec 1 za $w_2$ (a samozřejmě jazyk $L_t$ za $L$) -- Redukce $R$ pro libovolný $L' \in NP$ a řetězec $w$
  simuluje v polynomiálním čase DTS $M$: $L(M) = L'$ na vstupu $w$ a pokud $M$ přijme, redukce $R$ smaže pásku a zapíše $w_1$. Pokud odmítne, zapíše $w_2$ (délka řetězců $w_1$, $w_2$ nezáleží na vstupu -- jsou pevně zvolené, $R$ lze tedy implementovat DTS s polyn. čas. složitostí). 
  A také platí $w\in L' \Leftrightarrow R(w) \in L$ (viz předchozí důkaz).
  
\end{enumerate}



\end{document}