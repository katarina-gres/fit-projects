\documentclass[a4paper,12pt]{article}

\usepackage[utf8]{inputenc}
\usepackage[czech]{babel}
\usepackage[IL2]{fontenc}
\usepackage[left=3cm,text={15cm, 23cm},top=3.5cm]{geometry}

\usepackage{graphicx}
\usepackage{tikz}

\usepackage{amssymb}
\usepackage{amsmath}
\usepackage{amsthm}
\usepackage{forest}
\usepackage[ruled,czech,linesnumbered,noline]{algorithm2e}
\usepackage{array}
\usepackage{multirow}
\usepackage{textcomp}
\usepackage{listings}

\usepackage{xpatch}
\xpretocmd{\algorithm}{\hsize=\linewidth}{}{}
\xpretocmd{\procedure}{\hsize=\linewidth}{}{}

\usepackage{enumitem}
%\usepackage[neverdecrease]{paralist}

\usepackage{pgf}
\usepackage{tikz}
\usetikzlibrary{arrows,automata}

\newcounter{counten}
\setcounter{counten}{1}

\newtheorem{theorem}{Tvrzení}[counten]
\newtheorem{corollary}{Důsledek}[counten]

\lstset{ %
  basicstyle=\footnotesize,
  numbers=left                    % where to put the line-numbers; possible values are (none, left, right)
}


\title{Úkol č. 4 do předmětu Složitost}
\author{Vojtěch Havlena (xhavle03)}
\date{}

\makeatletter
\newcommand{\dotminus}{\mathbin{\text{\@dotminus}}}

\newcommand{\@dotminus}{%
  \ooalign{\hidewidth\raise1ex\hbox{.}\hidewidth\cr$\m@th-$\cr}%
}
\makeatother

\begin{document}

\maketitle

\setlist[enumerate,1]{leftmargin=0.2cm}

\begin{enumerate}[label=\textbf{\arabic*}.]

 \item {\bfseries Příklad}\\
  V případě, že uzel $max$ má stupeň $k$, tak největší klika v grafu může být velikosti $k+1$, což nastane
  tehdy, když uzel $max$ je součástí právě této největší kliky. Dále uvažujme následující třídu grafů 
  $\{G_n | n\in\mathbb{N}, n \geq 2\}$, kde $G_n$ je graf, který obsahuje uzel $u$ stupně $n$ (stupně 
  ostatních uzlů jsou ostře menší než $n$) a zároveň největší klika, kterou je uzel $u$ součástí má velikost 2. 
  Dále největší klika, která v grafu existuje má velikost $n$ (*příklad takového grafu). 
  
  Pokud na graf $G_n$ použijeme algoritmus $klika$, dostaneme velikost 2 (i když velikost největší kliky je $n$).
  Nyní tedy předpokládáme, že algoritmus $klika$ je $\epsilon$-aproximační pro nějaké $0 \leq \epsilon < 1$.
  Potom z definice $\epsilon$-aproximačního algoritmu dostáváme
  $$
    \frac{OPT(G_n) - |klika(G_n)|}{OPT(G_n)} = \frac{n - 2}{n} = 1 - \frac{2}{n} \leq \epsilon,
  $$
  což je ale spor, protože výraz $1 - \frac{2}{n}$ se pro rostoucí $n$ blíží k 1 (limita výrazu je 1) a 
  tudíž tento výraz není možné omezit nějakým pevným $\epsilon < 1$. Algoritmus $klika$ tedy není $\epsilon$-aproximační
  pro žádné $\epsilon < 1$.
  
 \item {\bfseries Příklad}\\
  Předpokládejme opak, tedy předpokládejme, že funkce $crypt$ je injektivní a $IMG_{crypt} \in NP\setminus UP$. Dále sestrojíme NTS $M$, který pro
  vstup $w \in IMG_{crypt}$ nedeterministiky uhodne $v \in \{0,1\}^*$ takové, že $w = crypt(v)$ (pokud takové $v$
  neexistuje, $M$ vstup zamítne). $M$ tedy pracuje tak, že nedeterministiky zvolí $v$ a potom na něm simuluje výpočet funkce $crypt$. 
  Platí tedy $L(M) = IMG_{crypt}$. Vzhledem k tomu, že $crypt \in FP$ -- je spočitatelná
  v polynomiálním čase, i celý NTS $M$ pracuje v polynomiálním čase. Navíc díky tomu, že $crypt$ je injektivní zobrazení,
  existuje pro NTS $M$ a vstup $w$ nejvýše jeden akceptační běh. A tím pádem $L(M) = IMG_{crypt} \in UP$.
  Což je ale spor s předpokladem. Platí tedy, že pokud $IMG_{crypt} \in NP\setminus UP$, potom $crypt$ není injektivní.
  
 \item {\bfseries Příklad}\\
  NC redukci provedeme následovně. Na vstupu předpokládáme bezkontextovou gramatiku $G = (N, \Sigma, P, S)$ a řetězec $w$.
  Redukce pracuje ve třech krocích:
  \begin{enumerate}
   \item K bezkontextové gramatice $G$ vytvoříme zásobníkový automat $P$ (přijímající s vyprázdněním zásobníku) takový,
     že $L(G) = L(P)$. Toto lze provést v konstantním čase pomocí $|P| + |\Sigma|$ procesorů v konstantním čase
     (každý procesor z pravidla gramatiky udělá přechod ZA a pro každý symbol abecedy se přidá přechod typu $\delta(q,a,a)$, 
     kde $q$ je poč. stav a $a\in\Sigma$). Dále k řetězci $w$ vytvoříme konečný automat $A$ přijímající právě řetězec $w$.
     K tomu stačí $|w|$ procesorů s konstantním časem (každý přidá jeden přechod). Pro celý tento krok je tedy potřeba $|P| + |\Sigma| + |w| \leq n$
     procesorů pracujících v konstantním čase.
   \item Provedeme průnik ZA $P$ a konečného automatu $A$. Výsledný zás. automat $P\times A$ bude mít $S_1\cdot S_2$ stavů (vzhledem k tomu, že
   ZA $P$ má jeden stav a $S_2$ $|w|$ stavů, platí $S_1\cdot S_2 \leq n$. Následně zjistíme přechodovou funkci. Každý procesor
   přidá nejvýše jeden přechod. Celkem je zapotřebí $(S_1\cdot S_2)\cdot (|\Sigma| + 1)\cdot |\Gamma| \leq n^3$ procesorů, které
   ovšem pracují v konstantním čase.
   \item Nakonec ZA $P\times A$ převedeme zpět na bezkontextovou gramatiku. Pro to je nutné nejprve zjistit délky pravých stran pravidel a 
   vygenerovat všechny posloupnosti stavů $P\times A$ o délce pravých stran pravidel $P$. Délka pravidel je konstanta nezávislá
   na vstupu, proto je toto možné provést v konstantním čase s využitím $O(n^c)$ procesorů ($c$ je největší délka pravé strany).
   Nakonec už jen vygenerujeme pravidla gramatiky podle algoritmu (slidy k předmětu TIN). K tomu je potřeba polynomiální počet procesorů $O(n^{c+3}) \sim n^{O(1)}$
   s konstantním časem.
  \end{enumerate}
 Vzhledem k tomu, jak je redukce $R$ navržena platí $(G,w) \in MEM \Leftrightarrow R(G, w) \in EMP$. Navíc $R$ je ve třídě $NC$,
 protože je pro výpočet třeba polynomiální počet procesorů pracujících v konstantním čase.
  
\end{enumerate}



\end{document}