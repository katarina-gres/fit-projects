\documentclass[a4paper,12pt]{article}

\usepackage[utf8]{inputenc}
\usepackage[czech]{babel}
\usepackage[IL2]{fontenc}
\usepackage[left=3cm,text={15cm, 23cm},top=3.5cm]{geometry}

\usepackage{graphicx}
\usepackage{tikz}

\usepackage{amssymb}
\usepackage{amsmath}
\usepackage{amsthm}
\usepackage{forest}
\usepackage[ruled,czech,linesnumbered,noline]{algorithm2e}
\usepackage{array}
\usepackage{multirow}
\usepackage{textcomp}
\usepackage{listings}

\usepackage{xpatch}
\xpretocmd{\algorithm}{\hsize=\linewidth}{}{}
\xpretocmd{\procedure}{\hsize=\linewidth}{}{}

\usepackage{enumitem}
%\usepackage[neverdecrease]{paralist}

\usepackage{pgf}
\usepackage{tikz}
\usetikzlibrary{arrows,automata}

\newcounter{counten}
\setcounter{counten}{1}

\newtheorem{theorem}{Tvrzení}[counten]
\newtheorem{corollary}{Důsledek}[counten]

\lstset{ %
  basicstyle=\footnotesize,
  numbers=left                    % where to put the line-numbers; possible values are (none, left, right)
}


\title{Úkol č. 3 do předmětu Složitost}
\author{Vojtěch Havlena (xhavle03)}
\date{}

\makeatletter
\newcommand{\dotminus}{\mathbin{\text{\@dotminus}}}

\newcommand{\@dotminus}{%
  \ooalign{\hidewidth\raise1ex\hbox{.}\hidewidth\cr$\m@th-$\cr}%
}
\makeatother

\begin{document}

\maketitle

\setlist[enumerate,1]{leftmargin=0.2cm}

\begin{enumerate}[label=\textbf{\arabic*}.]

 \item {\bfseries Příklad}\\
  Nejprve důkaz, že $MEM_1 \in PSPACE$. Sestrojíme NTS $M$, který má na vstupu $(M_L, w)$ a simuluje $M_L$ na vstupu $w$. 
  Vzhledem k tomu, že LOA je speciální případ nedeterministického TS, platí že $L(M) = MEM_1 \in NSPACE(n)$,
  A tedy podle Savitchova tvrzení, $MEM_1\in PSPACE$.
  
  Důkaz PSPACE těžkosti provedeme redukcí z $QBF$. Neformálně ke každé $QBF$ $\phi$ sestrojíme LOA $M_1$, který 
  bude simulovat rekurzivní vyhodnocování procedury $Eval$ (viz. přednáška PSPACE) a přijme právě když formule $\phi$ je pravdivá. 
  Vstupní páska bude reprezentovat zásobník pro simulaci $Eval$. Vzhledem k tomu, že hloubka zanoření je 
  omezena lineárně vzhledem k velikosti $\phi$, velikost pásky bude polynomiální k $|\phi|$ (počet kvantifikátorů plus počet operátorů). Počáteční řetězec 
  na vstupu je tedy nutné zvolit tak, aby jsme obsáhli max. velikost zásobníku.
  
  Více detailněji. Dále uvažuji, že $\phi \equiv Q_1x_1\dots Q_mx_m F$ je QBF. Nejprve si připravíme LOA $M_1$, který bude řídit výpočet procedury $Eval$.
  LOA $M_1$ navrhneme tak, aby pracoval nezávisle na formuli $\phi$. V průběhu výpočtu $M_1$ 
  obsahuje na pásce řetězec ve tvaru 
  $$\#\langle F\rangle\#(\#f x_1\dots x_m)^m\# d\#\#x'_1\dots x'_m\#\langle F'\rangle\#,$$
  kde $m$ je počet proměnných ve formuli $\phi$, 
  $x_1 \dots x_m$ je ohodnocení proměnných $x_1\dots x_m$, $f$ je příznak, který označuje, 
  jak je aktuálně sledovaná proměnná kvantifikována, $d$ je aktuální hloubka zanoření, $\langle F\rangle$ je 
  kód vstupní Booleovské formule a $x'_1\dots x'_m$ je aktuální ohodnocení proměnných pro které se má vyhodnotit 
  logická funkce $F$. 
  
  LOA $M_1$ pracuje potom následovně: Vstupní obsah pásky je 
  \begin{equation}
    \#\langle F\rangle\#\#\langle Q_1\rangle 0^m\#\dots \langle Q_m\rangle 0^m\#\langle m\rangle\#\#0^m\#\langle F\rangle\#
  \end{equation}
  (všechny 
  proměnné jsou nastaveny na hodnotu 0 a hloubka na $m$). LOA $M_1$ potom simuluje výpočet LOA $M_F$ (vyhodnocení funkce $F$ podle 
  zadaného přiřazení, popis dále)
  pro ohodnocení proměnných v hloubce $m$ -- za $x'_1\dots x'_m$ se nahradí ohodnocení z úrovně $m$ a 
  řetězec $\langle F'\rangle$ je obnoven za $\langle F\rangle$ (LOA $M_F$ přepíše řetězec reprezentující formuli, proto je nutné ho obnovit)
  Podle výsledku $M_F$ se informace propaguje o úroveň níž (řekněme do $i$), kde podle předaného výsledku,
  aktuální hodnoty $f_i$ a hodnoty $x_i$ provede akce podle procedury $Eval$ (buď se hodnota propaguje do další úrovně nebo je
  hodnota $x_i$ změněna na další hodnotu, pokud je to ovšem možné, a hodnoty všech proměnných $x_j$, $i < j \leq m$ na všech 
  vyšších hloubkách jsou nastaveny na 0 a celý postup je opakován). Pokud se informace dopropagovala až do první úrovně
  a procedura $Eval$ je již vyčíslená (rekurzivně jsme již získali odpověď na to, jestli $\phi$ je pravdivá), přijmeme 
  právě tehdy, když získaná hodnota je 1 ($\phi$ je pravdivá). Tedy $M_1$ přijmě právě tehdy, když vstupní formule $\phi$ je pravdivá.
  
  
  %Pokud $F$ je pro toto ohodnocení splnitelná, zkontroluje
  %se, zda je $x_m$ kvantifikována existenčně. Pokud ano, snížíme hodnotu hloubky a posuneme se 
  %na předchozí položku v zásobníku (na pásce). Zde opět kontrolujeme, zda je kvantifikována existenčně 
  %(případně se opět posuneme výše až do hloubky 1). Pokud je nějaká proměnná $x_i$ kvantifikována univerzálně,
  %a hodnota té proměnné je nastavena na 0, nastavíme ji na 1, ve všech následujících položkách na zásobníku
  %nastavíme hodnoty proměnných $x_j$, $i < j \leq m$ opět na 0 a hloubku nastavíme na $m$ a pokračujeme od začátku.
  
  %Pokud $F$ je pro toto ohodnocení není splnitelná, postup bude podobný, jako v předchozím případě, jen
  %se směrem do nižších hloubek propaguje hodnota 0.
  
  %LOA $M_1$ funguje obecně pro jakoukoliv formuli, lze si ho tedy předem připravit. Nicméně činnost stroje $M_F$
  %je závislá na konkrétní podobě formule $\phi$. 
  
  LOA $M_F$ na své vstupní pásce očekává vstupní řetězec ve tvaru $x_1\dots x_m\#\langle\phi\rangle$, kde
  $\langle\phi\rangle$ je kód formule $\phi$ a $x_1\dots x_m$ jsou ohodnocení proměnných $x_1\dots x_m$.
  LOA $M_F$ po ukončení výpočtu zapíše na pásku hodnotu $\phi(x_1\dots x_m)$. Ve zkratce, $M_F$ pracuje tak,
  že nejprve nahradí všechny výskyty proměnných ve řetězci $\langle\phi\rangle$ za jejich hodnoty. LOA $M_F$
  začne procházet vstupní řetězec s formulí a začne vyhodnocovat formuli od nejvnitřnějších podformulí a tyto 
  podformule nahrazuje za jejich výsledné hodnoty. (tedy např. nejprve se vyhodnotí podformule typu $\neg x_i$, 
  $x_i \wedge x_j$ atd. a potom teprve složitější podformule).
  Časová složitost tohoto přístupu bude velká, ale na druhou stranu je potřeba pouze ta část pásky, kde
  je zapsán vstup (tedy $M_F$ je opravdu LOA). Činnost $M_F$ je opět nezávislá na vstupní formuli~$\phi$.
  
  Hloubka rekurze procedury $Eval$ je $m$, kde $m$ je počet proměnných v $\phi$. Tedy pokud je vstupní řetězec
  zadán podle (1), stačí pro vyhodnocení pravdivosti QBF $\phi$ pouze prostor pásky, kde je zapsán vstup.
  Tedy LOA $M_1$ pro libovolnou QBF $\phi$ přijme právě když je pravdivá.
  
  Redukce $R$ tedy pro libovolnou QBF $\phi = Q_1x_1\dots Q_mx_m F$ vrátí dvojici $(M_1, w)$, kde 
  $$
    w = \#\langle F\rangle\#\#\langle Q_1\rangle 0^m\#\dots \langle Q_m\rangle 0^m\#\langle m\rangle\#\#0^m\#\langle F\rangle\#.
  $$
  Velikost řetězce $\langle F\rangle$ je $O(n^2)$, kde $n=|\phi|$ a ve stejném čase může být řetězec sestaven.
  Vzhledem k tomu, že $m\leq n$, je celková délka $|w| \in O(n^2)$ a tento řetězec může být sestaven v pol. čase.
  Tedy celá redukce $R$ může být implementována DTS v polynomiálním čase.
  Navíc platí:
  $$
    \phi \in QBF \Rightarrow R(\phi) \in MEM_1\quad \mbox{ a }\quad\phi \not\in QBF \Rightarrow R(\phi) \not\in MEM_1.
  $$
  Tedy $MEM_1$ je $PSPACE$-úplný.
  
 \item {\bfseries Příklad}\\
  Nejprve dokážeme, že $OPT\_PARTITION \in NPO$.
  \begin{itemize}
   \item[--] Množina vstupních případů problému je množina dvojic $(S, v)$, kde $S$ je množina položek a $v$ je váhová funkce.
    Deterministický Turingův stroj nejprve zkontroluje, zda na vstupu je správně zformovaná instance problému. Následně ověří,
    jestli je váhová funkce dobře definována (pro každou položku z $S$ zkontroluje, zda váhová funkce přiřazuje této položce 
    nějakou váhu). Tento TS pracuje v polynomiálním čase. Tedy $I\in P$.
   \item[--] Uvažujme nějaké $x\in I$. Potom pro každé $A\in F(x)$ platí $A\subseteq S$, tedy $|A| \leq |S|$. Přípustná řešení
    jsou tedy ohraničena polynomem.
   \item[--] Opět uvažujme nějaké $x\in I$. Pokud máme nějaké $|A|\leq |S|$, potom lze v polynomiálním čase (vzhledem k $|S|$) rozhodnout, zda
    $A\subseteq S$ (pro každý prvek z $A$, zkontrolujeme, jestli je i v $S$).
   \item[--] Cena řešení $c$, která je dána jako rozdíl součtu vah položek $S\setminus A$ a $A$ lze opět 
    vypočítat DTS v polynomiálním čase vzhledem k $|S|$ (pro každý prvek z $S\setminus A$ najdeme odpovídající váhy a sečteme je, 
    to samé pro $A$ a nakonec obě hodnoty odečteme.
  \end{itemize}
  V dalším kroku ukážeme, že jazyk asociovaný k tomuto opt. problému $L_{OP}$ je $NP$-úplný. Vzhledem k tomu, že $OPT\_PARTITION \in NPO$,
  tak $L_{OP} \in NP$. Důkaz, že $L_{OP}$ je $NP$-těžký provedeme redukcí z problému $PARTITION$. Nechť $(T,v)$ je instance problému
  $PARTITION$. Redukce $R$ pouze vstup transformuje na $((T,v), 0)$. Cena řešení je 0 právě když lze množinu $T$ rozdělit na dvě 
  disjunktní množiny se stejnou váhou. Potom tedy platí, že
  $$
    (T,v)\in PARTITION \Leftrightarrow R((T,v)) \in L_{OP}.
  $$
  Redukci lze zřejmě implementovat pomocí DTS v pol. čase. Tedy $L_{OP}$ je $NP$-úplný jazyk.
  
  Nyní už můžeme přejít k samotnému důkazu neexistence absolutního aproximačního algoritmu. Důkaz povedeme sporem, tedy předpokládáme, že 
  existuje absolutní aproximativní algoritmus $A$ s absolutní chybou $k$. Pomocí algoritmu $A$ vytvoříme polynomiální algoritmus, který 
  pro instanci $OPT\_PARTITION$ nalezene jeho optimální řešení. Tento polynomiální algoritmus nejprve pro vstupní případ $x = (S,v)$ 
  problému $OPT\_PARTITION$ sestrojí vstup $x' = (S, v')$, kde $v'(z) = (k+1)v(z)$, pro každé $z \in S$ a na vstupu $x'$ odsimuluje 
  výpočet algoritmu $A$. Přípustná řešení jsou pro oba případy stejná, tedy $F(x) = F(x')$ (výběr podmnožin $S$). Avšak cena přípustných 
  řešení problému $x'$ je $(k+1)$ násobek cen řešení~$x$.
  
  Algoritmus $A$ vypočítá $A(x')$. Vzhledem k tomu, že $A$ je absolutní apr. algoritmus s chybou $k$, platí:
  \begin{equation}\label{eq:2.1}
    |c'(A(x')) - OPT(x')| = \left|\left|\sum_{z\in S\setminus A(x')}v'(z) - \sum_{z\in A(x')}v'(z)\right| - OPT(x')\right| \leq k
  \end{equation}
  Vzhledem k tomu, že $v'(z) = (k+1)v(z)$, platí i $OPT(x') = (k+1)OPT(x)$. Tedy úpravou \eqref{eq:2.1} dostáváme
  \begin{equation}\label{eq:2.2}
    \left|\left|\sum_{z\in S\setminus A(x')}(k+1)v(z) - \sum_{z\in A(x')}(k+1)v(z)\right| - (k+1)OPT(x)\right| \leq k
  \end{equation}
  Celou nerovnici můžeme podělit $k+1$ a protože pracujeme v celých číslech dostáváme následující
  $$
    \left|\left|\sum_{z\in S\setminus A(x')}v(z) - \sum_{z\in A(x')}v(z)\right| - OPT(x)\right| = |c(A(x')) - OPT(x)| = 0.
  $$
  Tedy $A(x')$ je optimálním řešením pro případ $x$. Vzhledem k tomu, že $A$ pracuje s polynomiální složitostí (definice), 
  celý algoritmus pro výpočet opt. řešení pracuje v polynomiálním čase. Tedy $OPT\_PARTITION\in PO$ a vzhledem k tomu,
  že asociovaný jazyk $L_{OP}$ je $NP$-úplný, dostáváme, že $NP = P$, což je spor s předpokladem. Absolutní apr. algoritmus 
  tedy neexistuje.
 
\end{enumerate}



\end{document}