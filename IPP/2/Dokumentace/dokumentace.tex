\documentclass[a4paper,10pt]{article}

\usepackage[utf8]{inputenc}
\usepackage[czech]{babel}
\usepackage[IL2]{fontenc}
\usepackage{times}
\usepackage{float}
\usepackage[left=3cm,text={15cm, 23cm},top=3.5cm]{geometry}

\begin{document}

\noindent
\textbf{Dokumentace úlohy XQR}: XML Query v Python 3 do IPP 2013/2014\\
\textbf{Jméno a příjmení}: Vojtěch Havlena\\
\textbf{Login}: xhavle03

\section{Analýza řešení}
Cílem zadané úlohy je implementace skriptu, který vyhodnocuje dotaz SELECT nad 
vstupem ve formátu XML. Výstup je možné mimo jiné ovlivňovat podmínkami v dotazu.
Při řešení jsou využity znalosti formálních jazyků, a to především bezkontextových jazyků
a syntaktické analýzy.

\section{Implementace}
Celý program je z logického hlediska rozdělen do několika spolupracujících celků. První z nich 
(\texttt{lex.py}) zodpovídá za lexikální analýzu dotazu. Druhý celek (\texttt{syntax.py})
provádí syntaktickou analýzu zhora dolů a zárověň vyhodnocuje dotaz. Modul \texttt{expr.py}
provádí vyhodnocování podmínek dotazu za podpory funkcí z modulu \texttt{xmloperation.py}.
Poslední celek (\texttt{xqr.py}) zodpovídá za zpacování vstupu a volání operací z 
jednotlivých modulů.

Prvním krokem při vykonávání skriptu je načítání a kontrola parametrů příkazové řádky pomocí
funkce \texttt{getopt}. Taktéž je prováděna kontrola existence a oprávnění přístupu k 
zadaným souborům. Nemůže chybět inicializace proměnných zodpovídajících za správné vyhodnocení
dotazu.

\subsection{Lexikální analýza}
Úvodní část zpracování dotazu je lexikální analýza, která je podstatná pro pozdější syntaktickou analýzu. 
Pro rozpoznávání lexémů byl implementován konečný automat. Mezi zpracovávané lexémy jsou zařazena 
jednotlivá klíčová slova, XML elementy, operátory a literály řetězce a celého čísla. Během 
lexikální analýzy probíhá také kontrola validity XML elementu, či správné formátování řetězce.

\subsection{Syntaktická analýza a vyhodnocení dotazu}
Pro kontrolu syntaktické správnosti dotazu byla zvolena analýza zhora dolů a pro kontrolu správnosti podmínek
syntaktická analýza zdola nahoru. Během analýzy zhora dolů dochází k postupnému načítání tokenů a rekurzivním
voláním funkcí reprezentující neterminály se kontroluje správnost zadání dotazu. Po načtení názvu zdrojového
elementu dochází k jeho nalezení v XML stromové struktuře (použit byl algoritmus DFS). Pro práci s XML soubory 
byla použita knihovna \texttt{ElementTree} V dalším kroku se pomocí vestavěné funkce \texttt{iter} naleznou 
všechny cílové elementy a uloží se do seznamu. Pakliže byla zadána klauzule WHERE, dochází k 
vyhodnocení podmínky.

Kontrola syntaktické správnosti podmínek a jejich následné vyhodnocení je prováděna SA zdola nahoru pomocí
precedenční tabulky (viz. Tabulka 1). V tomto tvaru je schopna vyhodnocovat i složitější podmínky, čímž bylo implementováno
rozšíření LOG. Pro jednoduchost jako proměnnou $i$ uvažujeme celou podmínku ve tvaru 
\texttt{<operand> <operátor> <operand>}. Kvůli umožnění vyhodnocování podmínek se do zásobníku (reprezentován 
seznamem) neukládá pouze typ tokenu, ale také seznam XML elementů, které vyhovují jednotlivým podmínkám. Díky tomu 
je možné k logickým spojkám přistupovat jako k množinovým operacím. Tedy např. při vyhodnocení
operátoru AND se provede průnik seznamů, které byly vyjmuty ze zásobníku a výsledný seznam spolu s 
typem tokenu je uložen zpět na zásobník. Podobně pro operátor OR se jedná o sjednocení seznamů a pro 
operátor NOT o rozdíl všech cílových seznamů a daného seznamu. Během této analýzy dochází také
ke kontrole typů operandů. 

\begin{table}
  \centering
  \begin{tabular}{|c|c|c|c|c|c|c|c|}
    \hline
    & AND & OR & NOT & ( & ) & $i$ & \$ \\
    \hline
    AND & \textgreater & \textgreater & \textless & \textless & \textgreater & \textless & \textgreater \\ \hline
    OR & \textless & \textgreater & \textless & \textless & \textgreater & \textless & \textgreater \\ \hline
    NOT & \textgreater & \textgreater & \textless & \textless & \textgreater & \textless & \textgreater \\ \hline
    ( & \textless & \textless & \textless & \textless & = & \textless & \\ \hline
    ) & \textgreater & \textgreater & & & \textgreater & & \textgreater \\ \hline
    $i$ & \textgreater & \textgreater & & & \textgreater & & \textgreater \\ \hline
    \$ & \textless & \textless & \textless & \textless & & \textless & \$ \\
    \hline
  \end{tabular}
  \caption{Precedenční tabulka pro vyhodnocování výrazů}
\end{table}


Co se týče vyhodnocování jednoduchých podmínek, postupně pro všechny dosud 
vyhovující elementy dochází nejprve k nalezení porovnávaného elementu/atributu a dle operátoru k příslušnému 
porovnání. Nevyhovující elementy jsou zahozeny. Pro správné porovnání musí také dojít k eventuálnímu 
přetypování hodnot elementů/atributů.

Po vyhodnocení všech podmínek je počet vyhodnocených elementů omezen parametrem klauzule LIMIT. 
Nakonec je výsledek dle vstupních parametrů uložen buď do zadaného souboru nebo vypsán na standardní výstup.
Předtím je ovšem provedeno ještě eventuální obalení výsledku kořenovým elementem či přidání XML hlavičky.

\end{document}
