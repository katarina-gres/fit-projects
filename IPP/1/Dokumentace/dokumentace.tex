\documentclass[a4paper,10pt]{article}

\usepackage[utf8]{inputenc}
\usepackage[czech]{babel}
\usepackage[IL2]{fontenc}
\usepackage{times}
\usepackage{float}
\usepackage[left=3cm,text={15cm, 23cm},top=3.5cm]{geometry}

\begin{document}

\noindent
\textbf{Dokumentace úlohy JMP}: Jednoduchý makroprocesor v PHP 5 do IPP 2013/2014\\
\textbf{Jméno a příjmení}: Vojtěch Havlena\\
\textbf{Login}: xhavle03

\section{Analýza řešení}
Cílem zadané úlohy je implementace skriptu, který plní funkci jednoduchého makroprocesoru.
Makroprocesor je mimo jiné schopen definovat a volat vlastní makra. Při řešení jsem 
využil znalosti formálních jazyků (zejména konečných automatů).

\section{Implementace}
Celý program jsem logicky rozdělil do tří spolupracujících celků. První celek (\texttt{lex.php})
zodpovídá za lexikální analýzu vstupního souboru, druhý celek (\texttt{semantic.php}) provádí expanzi maker
a kontrolu syntaktické správnosti vstupního souboru. Poslední celek (\texttt{jmp.php}) kontroluje vstupní
parametry a zodpovídá za správné volání funkcí z jednotlivých celků.

První úkon, který program provádí je načítání a kontrola parametrů. Načítání je prováděno pomocí 
funkce \texttt{getopt}. V této fázi je také zkontrolována případná neexistence 
vstupního souboru či třeba chybné kódování vstupního souboru. Dále je provedena inicializace
globálních proměnných (tabulka maker) a nakonec samotné zpracovávání vstupu.

\subsection{Zpracovávání vstupu}
Pro zpracovávání vstupu jsem se rozhodl implementovat jednoduchý lexikální analyzátor.
Mezi rozpoznávané lexémy jsem zařadil znak, název makra, identifikátor proměnné a 
závorky. Vzhledem k požadavkům úlohy nebylo možné rozpoznávání lexémů řešit pouze jediným
univerzálním konečným automatem, ale bylo nutné implementovat dva kon. automaty, které 
rozpoznávali lexémy v bloku a mimo blok. Během rozpoznávání je také prováděna náhrada
escape sekvencí. 

Čtení ze vstupu probíhá po jednotlivých znacích. Pokud je nutné 
určitý text poslat opětovně na vstup pro znovuzpracování (rozgenerovávání maker),
využívá se řetězce, ze kterého jsou jednotlivé znaky načítány prioritně. To znamená, že
pokud je řetězec, kam je uložen znovuzpracovávaný text neprázdný, další znak bude 
přečten právě z tohoto řetězce.

\subsection{Definice a volání maker}
Proces generování nového makra pomocí \texttt{@def} spočívá nejprve v načtení názvu a seznamu argumentů makra.
V případě, že je program spuštěn v módu neumožňující redefinici maker, provede se kontrola, 
zda makro není již uloženo v tabulce maker. Načtený seznam argumentů je uložen do tabulky maker.
Dále je načteno samotné tělo makra, které je uloženo do tabulky maker. Před uložením
se z celého těla odstraní výskyty argumentů, které se vyskytují i v načteném seznamu argumentů.
Pro každý odstraněný argument se zapamatuje pozice, na kterých se nacházel pro pozdější
volání maker se skutečnými parametry. 

V případě definování nového makra pomocí maker s funkcí \texttt{@let} se nejprve
načtou názvy obou maker (zdrojové a cílové makro). V dalším kroku se kontroluje existence
zdrojového makra v tabulce maker a zda neredefinujeme makra, která nemohou být redefinována.
Nakonec se podle cílového makra provede buď kopie záznamu v tabulce maker nebo se makro v 
tabulce zruší.

Proces volání makra spočívá nejprve v kontrole, zda je dané makro definováno. Následně je 
provedeno načtení skutečných argumentů (počet argumentů je uložen v tabulce maker). Dále se 
projde uložené tělo volaného makra znak po znaku a pro každou pozici se kontroluje, zda se 
sem nemá vložit nějaký načtený argument. Po vložení načtených argumentů se zpracované tělo pošle 
opět na vstup pro znovuzpracování lexikálním analyzátorem. Při volání makra, kterému byla přiřazena funkce
vestavěného makra (například \texttt{@def}, \texttt{@set} apod.) se zavolají specializované funkce.

\subsection{Tabulka maker}
Tabulka maker je implementována jako asociativní pole, kde klíč je název makra a hodnota je instance
třídy \texttt{TableItem}. Tato třída obsahuje informace o počtu parametrů, těle makra, seznamu jednotlivých
parametrů a jejich umístění v těle makra. 

Umístění argumentů je řešeno opět jako asociativní pole, kde klíč 
odpovída pozici v těle a hodnota argumentu, který se má na danou pozici vložit. Dále třída obsahuje položku, 
která nese informaci o tom, zda má makro přiřazenu funkci vestavěného makra (tato funkcionalita může být 
přiřazena například makrem \texttt{@let}). 

Na začátku programu je provedena inicializace tabulky maker vložením vestavěných maker.



\end{document}
